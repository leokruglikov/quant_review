\documentclass[12pt]{article}
\usepackage[utf8]{inputenc}
\usepackage{graphicx} % Allows you to insert figures
\usepackage{amsmath} % Allows you to do equations
\usepackage{fancyhdr} % Formats the header
\usepackage{geometry} % Formats the paper size, orientation, and margins
\setlength{\parindent}{0pt} % no paragraph indents
\setlength{\parskip}{1em} % paragraphs separated by one line
\usepackage[format=plain,
            font=it]{caption} % Italicizes figure captions
\usepackage[english]{babel}
\usepackage{csquotes}
\renewcommand{\headrulewidth}{0pt}
\geometry{letterpaper, portrait, margin=1in}
\setlength{\headheight}{14.49998pt}
\usepackage{mathtools}
\usepackage[normalem]{ulem}
\usepackage{caption}
\usepackage{subcaption}
\usepackage{cite}
\usepackage{nicefrac}
\usepackage{wrapfig}
\usepackage{hyperref}
\usepackage[all]{xy}
\usepackage{qcircuit}
\usepackage{amsthm}
\usepackage{physics}
\usepackage{url}
\usepackage{hyperref}
\usepackage{filecontents}
\usepackage[backend=bibtex]{biblatex}
\usepackage{xcolor}
\usepackage{unicode-math}
\usepackage{tabularray}
\usepackage{bbm}
\usepackage{makecell}
\usepackage{booktabs}
\usepackage{amsfonts}
\addbibresource{./q.bib}



\title{Basic review of basic quantum computing algorithms}
\begin{document}
\maketitle

\newcommand{\qvdots}{% \vdots for qcircuit
  \raisebox{0.3em}{\ensuremath{\vdots}}%
}



\theoremstyle{concept}
\newtheorem{concept}{Concept}[section]




\section{Concepts and notions}

\subsection*{Fundamentals}
\paragraph{} Here I non-rigourously describe the main notions that are used during the algorithm discussions.
\begin{concept}
  A quantum cirquit is composed of a (multiple) qubit(s) input and a measurable output of the same size.
  If multiple qubits are involved, the inputs are mathematically defined: 
  \begin{gather}
    \text{input of n qubits }\ket{\psi_i} = \bigotimes_{i=1}^{n}\ket{\psi_i}
  \end{gather}
\end{concept}

\begin{concept}
  A quantum circuit is composed of different and various gates, that can be applied on multiple qubits. Every gate is an operator.
  If a gate $\widehat{O}$ is applied only on state $\ket{\psi_i}$, the operator on the whole system is given by 
  \begin{gather}
    \widehat{O}_{\text{on state i}} = \mathbbm{1}_{(1)}\otimes \mathbbm{1}_{(2)}\otimess \cdots \otimes \mathbbm{1}_{(i-1)}\otimes 
    \widehat{O}\otimes \mathbbm{1}_{(i+1)} \cdots \otimes \mathbbm{1}_{(n)}
  \end{gather}
\end{concept}
%\begin{concept}[Quantum function evaluation]
%Consider some kind of quantum cirquit and a function. I define the \textsl{(basic)} quantum function evaluation by the circuit given below:
%\end{concept}
%
%
%\begin{figure}[ht!]
%  \Qcircuit @C=1em @R=1em{
%    \lstick{\ket{0}} & \qw & \multigate{1}{U} & \qw \\
%    \lstick{\ket{0}} & \qw & \ghost{U} & \qw \\
%    \lstick{\ket{0}} & \qw & \ghost{U} & \qw \\
%  }
%  \label{fig:function_evaluation}
%\caption{Quantum function evaluation.}
%\end{figure}
%
%What the binary algorithm does is to \textsl{evaluate the function of the target cubit}.

\subsection*{Main Circuits}
Let's discuss main representation of the circuits.

\begin{table}[ht!]
  \begin{tabular}{|c|c|c|c|c|}
    \hline
    Operator & \makecell{Bra-Ket \\ representation} & Matrix & \makecell{Other math \\ properties} & Qiskit \\
    \hline 
    \hline
    \makecell{Identity op. \\ $I=\text{Id}=\mathbbm{1}$}  & \makecell{Closure relation: \\ $\forall \{a_i\}\in V$ \\ $\mathbbm{1}=\sum_i \ket{v_i}\bra{v_i}$ } & $\begin{bmatrix} 1 & 0 \\ 0 & 1 \end{bmatrix}$ & $\forall \ket{\psi} \in V , \mathbbm{1}\ket{\psi}=\ket{\psi}$ &  \\
    \hline
    \makecell{Hadamard op. \\ $H=\mathcal{H}$ } & \makecell{In the $Z/X$ basis \\ $\mathcal{H}=\ket{+}\bra{0} + \ket{-}\bra{1}$} &
    $\frac{1}{\sqrt{2}}\begin{bmatrix} 1 & 1 \\ 1 & -1 \end{bmatrix}$ & \makecell{Change of basis from \\ $Z$ to $X$ basis.} & \\
    \hline 
    \makecell{Z-gate \\ $\sigma_z = Z$} & $\sigma_z \equiv \ket{0}\bra{1} - \ket{1}\bra{1}$ & $ \begin{bmatrix} 1 & 0 \\ 0 & -1 \end{bmatrix} $ &
    \makecell{Eigen-op. of the \\ $\ket{+1;\hat{S}_z}=\ket{0}$ state} & \\
    \hline
    \makecell{X-gate \\ $\sigma_x = X$} & \makecell{$\sigma_x \equiv \ket{0}\bra{1}+\ket{1}\bra{0}$ \\ $\sigma_x \equiv \ket{+}\bra{+} - \ket{-}\bra{-}$ } & $\begin{bmatrix} 0 & 1 \\ 1 & 0 \end{bmatrix}$ & 
    \makecell{Eigen-op. of the \\ $\ket{+}\equiv \frac{1}{\sqrt{2}}(\ket{0}+\ket{1})$} & \\
\hline
    \makecell{phase gate\\ $P(\phi)$} & $P(\phi)\equiv \ket{0}\bra{0} + e^{i\phi}\ket{1}\ket{1}$ & $\begin{bmatrix} 1&0\\0&e^{i\phi} \end{bmatrix}$ & & \\
    \hline
    \makecell{Controlled-NOT \\
      \Qcircuit @C=1em @R=0.5em{
    \qw & \ctrlo{1}&\qw\\
    \qw & \targ & \qw
    }
  }& \makecell{
  $\text{CNOT}=\\=\ket{0}\bra{0}\otimes \mathbbm{1}+\ket{1}\bra{1}\otimes X$\\
  \ket{x_{c}}\ket{x_{t}}\xrightarrow{CNOT} \ket{x_{c}}\ket{x_{c}\oplus x_{t}}
  } & 
  $\begin{bmatrix}
    1&0&0&0\\
    0&1&0&0\\
    0&0&0&1\\
    0&0&1&0
  \end{bmatrix}$ &
  \makecell{Acting on a 4x1 vec.\\spanned by \\$\{\ket{00},\ket{01},\ket{10},\ket{11}\} } 
  & \\
  \hline
  \end{tabular}    
\end{table}

\newpage

\theoremstyle{definition}
\newtheorem{definition}{Definition}[section]

\section{Deutsch-Jozsa algorithm}
The Deutsch-Jozsa algorithm was the first algorithm to be proposed.
The goal of the Deutsch's quantum algorithm is to determine a concrete property of a funciton. In this case, we use the quantum 
algorithm, to determine the properties of the function. In the case of the 
\begin{definition}[Constant and balanced]
  The function used here is a function from $\{0,1\}^N \mapsto \{0,1\}$. The outputs of it can be either constant or balanced.
  We say that the function is \textit{constant} if for \underline{any} input the output is constant, that is, whether all ones or all zeroes.
  We say that the function is \textit{balanced} if it outputs ones half of the times and zeroes half of the other times (that is, out of all 
possible inputs, it will output a \verb|1| in the half of the input cases and \verb|0| in the other).
\end{definition}
We will consider the case where $n$ (the number of inputs) is equal to one. That is, the function $f(x)$ has one input bit. $f(x)$, with $x \in \{0,1\}$.
Let's see examples of functions that are either balanced or constant. Let $f_1(x)$. The function happens to output the following: $f_1(0)=0$ and $f_1(1)=0$. We see that the 
function has zeroes for any input. Therefore, we can conclude that the function $f_1$ is constant. Consider now the function $f_2$, satisfying the following:
$f_2(0)=1$ and $f_2(1)=0$. As there are 2 kinds of inputs (\verb|0| and \verb|1|), there are only 2 possible outputs (2 different outputs for 2 different inputs).
We see that for the function $f_2$, half of the outputs (i.e. only one out of two) is \verb|0| and half of the inputs is \verb|1|. Therefore, the function $f_2$ is 
balanced.

The example was a simple case, where the input was simply either \verb|0| or \verb|1|. The definition of this constant/balanced can be extended to a more complex case 
of an input $\{0,1\}^n$, which is the Deutsch-Jozsa algorithm. 
\subsection*{Deutsch's algorithm}
Let's first consider the Deutsch's algorithm \cite{noauthor_deutschjozsa_2022}, that is, with one input bit. Note that in this algorithm we assume or we are promised that the given function is either balanced or constant (it can be neither). The circuit is \textit{postulated} to be.
In order for it to have a working mechanism, we need to prepare the initial states.


\begin{figure}[ht!]
     \begin{subfigure}[b]{0.4\textwidth}
\Qcircuit @C=1em @R=1em{
  \qw & \gate{\mathcal{H}} & \ctrl{1} & \gate{\mathcal{H}} & \qw \\
  \qw & \qw      & \gate{f} & \qw      & \qw \\
}
\caption{Deutsch's circuit}
     \end{subfigure}
     \hfill
     \begin{subfigure}[b]{0.4\textwidth}
\hspace{1cm}
\Qcircuit @C=1em @R=1em{
  \lstick{\ket{0}}           & \qw & \gate{\mathcal{H}}& \ustick{A}\qw  & \ctrl{1} &\ustick{B}\qw & \gate{\mathcal{H}} & \qw \\
  \lstick{\ket{0} - \ket{1}} & \qw & \qw     &\ustick{A}\qw   & \gate{f} & \qw          &\qw      & \qw \\
}

\caption{Initialized states}
     \end{subfigure}
  \caption{Deutsch's circuit}
  \label{cirq:deutsch_start}

\end{figure}

Having the circuit with prepared initial states (\autoref{cirq:deutsch_start}) we're now in the situation in carefully initialize the cirquit. We've identified 2 states of the cirquit: \verb|A| and 
\verb|B|, which we'll quickly discuss \footnote{The $\ket{0}-\ket{1}$ state can be created using the Hadamard gate applied on the \ket{0} state. TODO}.

The first register begins with the state $\ket{0}$. As usual, applying the Hadamard gate $ \mathcal{H} $ on the first register gives us 
$\ket{0} \xrightarrow{\mathcal{H}}{} \frac{1}{\sqrt{2}} ( \ket{0} + \ket{1} )$ as usual. Thus, the total state at the point \textbf{A} is given by the tensor product of both registers, which is $(\ket{0}+\ket{1})\otimes (\ket{0} - \ket{1})$.

After the point \texbf{A} we apply the \textsl{function evaluation} (TODO). Let's consider the rigourous operation:


\begin{math}
  \text{\textmd{state at \textbf{A}}:\ \ \ } \frac{1}{\sqrt{2}}(\ket{0} + \ket{1}) \otimes \frac{1}{\sqrt{2}}(\ket{0}-\ket{1}) \\ 
\end{math}

Now, the state after the function $f$, using the concept of the evaluation function, we have the \textit{general} expression of $\ket{x} \otimes \ket{y} \mapsto \ket{x}\otimes \ket{y \oplus f(x)}$, 
which, in the case of our circuit \autoref{cirq:deutsch_start}, gives the state $\ket{y}=\frac{1}{\sqrt{2}}(\ket{0} - \ket{1}) $ for $\ket{y}$.
Thus, for the circuit between \textbf{A} and \textbf{B}:
\begin{align}
\ket{\psi_B}=\bigg| \frac{1}{\sqrt{2}} (\ket{0}+\ket{1}) \bigg \rangle  &\otimes \bigg| f(x) \oplus \frac{1}{\sqrt{2}} (\ket{0}-\ket{1}) \bigg \rangle \\
\frac{1}{2} \bigg| \ket{0}+\ket{1} \bigg \rangle  &\otimes \bigg| \ket{f(x)}-\ket{1\oplus f(x)} \bigg \rangle \\
\frac{1}{2} \bigg| \ket{0}+\ket{1} \bigg \rangle  &\otimes \bigg| (-1)^{f(x)} (\ket{0}-\ket{1}) \bigg \rangle
\label{eq:deutsch_f(x)}
\end{align}

Now we can expand this further, still for the state at the point B. When we explicit the expression 
in order to see the input of the function $f(x)$. What I mean is that in \autoref{eq:deutsch_f(x)}, the function 
$f(x)$ it accepts the superposition, as the input is an entangled state.

\begin{align}
\ket{\psi_B} = \frac{1}{2} \bigg| \ket{0}+\ket{1} \bigg \rangle  \otimes \bigg| (-1)^{f(x)} (\ket{0}-\ket{1}) \bigg \rangle \\
\ket{\psi_B} = \frac{1}{2} \bigg| (-1)^{f(x)}(\ket{0}+\ket{1})\bigg \rangle \otimes \bigg| \ket{0}-\ket{1} \bigg \rangle \\
\ket{\psi_B} = \frac{1}{2} \bigg| (-1)^{f(0)}\ket{0} + (-1)^{f(1)}\ket{1} \big \rangle \otimes \bigg| \ket{0} - \ket{1} \bigg \rangle
\end{align}
now we apply an unusual trick is not that simple to see but simple to check that it is true. What we will use here is the fact that 
\begin{align}
  f(0)\oplus f(0) \oplus f(1) = f(1)
\end{align}
This can be seen as the term $f(0) \oplus f(0)$ will always give zero and thus, $0 \oplus f(x)=f(x)$ and therefore, 

\begin{align}
\ket{\psi_B} = \frac{1}{2} \bigg| (-1)^{f(0)}\ket{0} + (-1)^{f(0)\oplus f(0)\oplus f(1)}\ket{1} \bigg \rangle \otimes \bigg| \ket{0} - \ket{1} \bigg \rangle \\
\ket{\psi_B} = \frac{1}{2} (-1)^{f(0)}\bigg| \ket{0} + (-1)^{f(0)\oplus f(1)}\ket{1} \bigg \rangle \otimes \bigg| \ket{0} - \ket{1} \bigg \rangle
\label{eq:deutsch_global_phase}
\end{align}
We know however that the global phase do not matter at all. In addition to that, we see that in \autoref{cirq:deutsch_start}, after the point \textbf{B},
we're not interested in the state $\ket{y}$ anymore. Thus, we can simply ignore both the global phase $(-1)^{f(0)}$ and the second state $\ket{0}-\ket{1}$.
Thus, we're left with the state $\ket{\tilde{\psi_B}}\equiv \ket{0} + (-1)^{f(0)\oplus f(1)}\ket{1}$, on which we apply the $\mathcal{H}$ operation.
This gives us 
\begin{align}
  \mathcal{H} \ket{\tilde{\psi_B}} \stackrel{*}{=} \ket{+} + (-1)^{f(0)\oplus f(1)}\ket{-}=\\
  (1+(-1)^{f(0)\oplus f(1)}) \ket{0} + (1-(-1)^{f(0)\oplus f(1)})\ket{1}
  \label{eq:deutsch_f_last}
\end{align}
Therefore, we obtain that if ${f(0)\oplus f(1)}=0$, the measurement of the top qubit will be $\ket{0}$ and if the result ${f(0)\oplus f(1)}=1$, the result of the 
measurement will be given by $\ket{1}$.
Therefore, we will measure the $\ket{0}$ bit if the value of the function is constant (either all ones or all zeroes), and similarly, we will measure the 
value $\ket{1}$ bit if the value of the function is balanced. In both cases, we're making the measurements with probability 1.

\subsection*{Deutsch-Jozsa's algorithm}
The Deutsch-Josza's algorithm is similar to the Deutsch's one. That is, it is a generalization from one bit input ( i.e. $\{0,1\}$) 
to the n bit input (i.e. $\{0,1\}^n$, which is nothing but a bit string).

\begin{table}[!hbt]
  \centering
\begin{tblr}{l || r}
\Qcircuit @C=1em @R=1em{
  \lstick{\ket{0}}&\qw  & \ustick{A}\qw & \gate{\mathcal{H}} & \ustick{B} \qw & \ctrl{1} & \ustick{C} \qw & \gate{\mathcal{H}}& \ustick{D}\qw \\
  \lstick{\ket{0}}&\qw  & \qw & \gate{\mathcal{H}} & \qw & \ctrl{1} & \qw & \gate{\mathcal{H}}& \qw \\
  \lstick{\ket{0}}&\qw  & \qw & \gate{\mathcal{H}} & \qw & \qw & \qw & \gate{\mathcal{H}}& \qw \\
  \lstick{\ket{0}}&\qw  & \qw & \qvdots & & \qvdots & & \qvdots & \\
  \lstick{\ket{0}}&\qw  & \qw & \gate{\mathcal{H}} & \qw & \ctrl{1} & \qw & \gate{\mathcal{H}}& \qw \\
  \lstick{\ket{0}}&\qw  & \qw & \gate{\mathcal{H}} & \qw & \ctrl{1} & \qw & \gate{\mathcal{H}}& \qw \\
  \lstick{\ket{0}-\ket{1}}&\qw & \qw & \qw & \qw & \gate{f} & \qw & \qw & \qw \\
  %\qw & \gate{\mathcal{H}} & \ctrl{1} & \gate{\mathcal{H}} & \qw \\
  %\qw & \qw      & \gate{f} & \qw      & \qw \\
}\hspace{1cm}&
\SetCell[r=9]{} 
\hspace{2cm}
\Qcircuit @C=1em @R=1.5em{
  \lstick{\ket{0}^{\otimes n}} \qw & \ustick{A}\qw & \gate{\Huge\mathcal{H}^{\otimes n}} & \ustick{B} \qw & \ctrl{1} & \ustick{C} \qw & \gate{\Huge\mathcal{H}^{\otimes n}} & \ustick{D}\qw \\
  \lstick{\ket{0}-\ket{1}} & \qw & \qw & \qw & \gate{f} & \qw & \qw & \qw \\
}

\\

\end{tblr}
\caption{The table illustrates 2 schematic circuits for the Deutsch-Josza algorithm. 
Both of them are equivalent. We prefer the second one (the right one) as it is simply shorter.}
\label{table:cirq:deutsch_josza}
\end{table}




The working principle of this generalized version is harder to show and harder to understand. The proof is nicely shown in \cite{noauthor_deutschjozsa_2022}.
Lets start from the initial state, as shown in \autoref{table:cirq:deutsch_josza}. As usual, we sometimes omit the global normalization constant 
(we thus write $\stackrel{*}{=}$).
\begin{align}
  \ket{\psi_A}\stackrel{*}{=}& (\ket{0}\otimes \ket{0} ... \otimes \ket{0})\otimes (\ket{0}-\ket{1}) = \ket{0}^{\otimes n}\otimes (\ket{0}-\ket{1})\\
  \ket{\psi_B}\stackrel{*}{=}& \bigotimes_{i=0}^{n} \mathcal{H} \ket{\psi_A} \stackrel{*}{=} \mathcal{H}^{\otimes n}\ket{0}^{\otimes n}\otimes (\ket{0}-\ket{1})\\
  \stackrel{*}{=}& \ket{+}^{\otimes n} (\ket{0}+\ket{1}) = \sum_{x}^{2^n-1} \ket{x}\otimes (\ket{0}-\ket{1})
\end{align}
Note that $\ket{x}$ here represents a binary string. Indeed, a $\mathcal{H}$ acting on $\ket{0}$ can represent the sum of numbers from 0 to 1. 
($\mathcal{H}\ket{x}\stackrel{*}{=}\ket{0}+\ket{1}$). When acting on a 2D space, 
$(\mathcal{H}\otimes \mathcal{H})(\ket{0}\otimes \ket{0})\stackrel{*}{=}\ket{0}\otimes \ket{0} + \ket{0}\otimes \ket{1} + \ket{1}\otimes \ket{0} + \ket{1}\otimes \ket{1}$.
We therefore see that the Hadamard gate on the n-dimensional space will give the sum of binary strings ranging from $0$ to $2^n-1$. This is what is meant 
in the sum over $\ket{x}$.
Then, the quantum function evaluation, as defined, will create the transformation $\ket{x}\otimes \ket{y} \xrightarrow{f}{} \ket{x}\otimes \ket{f(x)\oplus y}$,
by definition.
Therefore, at the point $C$ (as on \autoref{table:cirq:deutsch_josza}), the transformation give:
\begin{align}
  \ket{\psi_C} &\xleftarrow{f}{} \ket{\psi_B} \\
  \ket{\psi_C} &\stackrel{*}{=} \sum_x^{2^n-1} \ket{x}(\ket{f\oplus 0}-\ket{f \oplus 1}) \\
               &= \sum_x^{2^n-1} \ket{x}(-1)^{f(x)} (\ket{0}-\ket{1})
\label{eq:deutsch_josza_C}
\end{align}
now, we can "ignore" the $\ket{0}-\ket{1}$ part, as after the C point, we're only applying the $\mathcal{H}$ to the top part (i.e. without the 
$\ket{0}-\ket{1}$). We also remember our important result 
\begin{align}
  \matchcal{H} \ket{x} \equiv \bigotimes^n \mathcal{H} \ket{x} = \frac{1}{\sqrt{2^n}} \sum_{y=0}^{2^n-1} (-1)^{x\cdot y}\ket{y}
  \label{eq:hadamard_on_generic}
\end{align}
with the $x \cdot y$ being the vector product between two bitstrings. The bitstrings can be represented as vectors, e.g. $\ket{x}\equiv (0,1,1,0,0,...,1)$ and 
$\ket{y}\equiv (1,0,1,1,...,1)$. Their dot product is defined as $x\cdot y = (x_1 \cdot y_1) \oplus (x_2 \cdot y_2) \oplus (x_3 \cdot x_3) \oplus ... \oplus (x_n \cdot x_n)$.
Therefore, by applying the $\mathcal{H}$ on $\ket{\psi_C}$ given in \autoref{eq:deutsch_josza_C},
\begin{align}
  \bigotimes ^n \mathcal{H} \ket{\psi_C} &\stackrel{*}{=} \mathcal{H} \sum_x^{2^n-1} \ket{x}(-1)^{f(x)} \\
                                         &= \sum_x^{2^n-1} \mathcal{H}\ket{x}(-1)^{f(x)} \stackrel{\text{\autoref{eq:hadamard_on_generic}}}{=} \\
                                         &= \sum_x^{2^n-1} \sum_y^{2^n-1} \ket{y} (-1)^{f(x)}(-1)^{x\cdot y} \\
                                         &= \sum_y^{2^n-1} \Biggl [\; \; \sum_x^{2^n-1} (-1)^{f(x)}(-1)^{x\cdot y} \; \Biggl ] \ket{y}
\end{align}
which is the (almost) final result. This is what we get, when we'll obtain after applying the $\mathcal{H}$ on the $\ket{\psi_C}$ state.
Now, we can see that the $\ket{y}$ are linear combinations of probability amplitudes $\sum_x^{2^n-1} (-1)^{f(x)}(-1)^{x\cdot y}$ for each 
$\ket{y}$. We now have a look at the state $\ket{y}=0$, that is, $\ket{y}=\ket{0,0,...,0}$. The probability of measuring it 
is given by $\bigg| \sum_x^{2^n-1} (-1)^{f(x)} \bigg|^2$. The probability of measuring this $\ket{0}=\ket{y}$ will be given by 1, if the function is constant.
Indeed, if $f(x)$ is constant (either always ones or always zeroes), then we will make the sum of $-1$ if it is always one, and sum of all $(-1)^0$ if always 
zeroes. If it is balanced, then we will make the sum of $(-1)$ and $1$, which will give us 0 probability if the function is balanced.

Thus, we've showed that the algorithm can determine, whether the function is constant or balanced.

\subsubsection*{Example}

The "problem" with this algorithm is that we need to have a very specific oracle. To be more precise, one may thing that we constructed the concept of 
solving the algorithm, based on the given oracle that "comes from nowhere". This is not false. For an arbitrary function, we can't construct the algorithm right 
away.


In order to ckeck the algorithm, one can simply check the algorithm for specific inputs, and whether it represents the specific outputs.




















\subsection*{Bernstein-Vazirani algorithm}
The next algorithm is the Bernstein-Vazirani algorithm, which also gives a 
speedup over the classical solution. The technique to solve such problem is similar to the 
Deutsch-Jozsa's one, that is, uses a phase kick-back trick.

The problem that the Bernstein-Vazirani algorithm aims to solve is the following: we have a function $f(x)$ which has the form 
$f(x): \{0,1\}^n \rightarrow \{0,1\}$ and $f(x): x \mapsto s \cdot x=s_1 x_1 \oplus s_2 x_2 \oplus ... \oplus s_n x_n$, for some 
"secret" string $s$, for which we denote the corresponding function $f_s(x)$. The goal of the Bernstein-Vazirani problem is 
to determine this "secret" string $s$.


\begin{wraptable}{l}{7cm}
  \begin{tblr}{r}
    \Qcircuit @C=1em @R=1em{
      \lstick{\ket{0}^{\otimes n}}& \qw & \gate{\mathcal{H}^{\otimes n}} & \qw & \ctrl{1} & \qw & \gate{\mathcal{H}^{\otimes n}} & \qw \\
      \lstick {\ket{-}}& \qw & \qw                & \qw & \gate{f} & \qw & \qw   & \qw               
    }
  \end{tblr}
\end{wraptable}
\label{cirq:bern_vazirani}

As usual, in order to implement this algorithm, we need a quantum oracle for $f_s(x)$. In this case, the quantum oracle will be 
defined as for the Deutsch-Jozsa's algorithm as $\ket{x}\otimes \ket{y} \xrightarrow{U_{f_s}}{} \ket{x}\otimes \ket{y\oplus f_s(x)}$.
Similar to the case of the Deutsch-Jozsa, the state that will get through the oracle will be $\ket{x}\otimes \frac{1}{\sqrt{2}}(\ket{0}-\ket{1})$.
The result will be given by $\ket{x}(-1)^{f_s(x)}$ as shown in \autoref{eq:deutsch_josza_C}, where we decided to "drop" the second part with 
$\ket{-}$. The circuit is given in \autoref{cirq:bern_vazirani}.

Let's now show mathematically the Bernstein-Vazirani algorithm:
\begin{align}
  \ket{0}^{\otimes n} \xrightarrow{\mathcal{H}^{\otimes n}} \sum_x^{2^n} \ket{x}
\end{align}
which is the state of the top-register before we reach the quantum function evaluation. Further we write:
\begin{align}
  \sum_x^{2^n-1} \ket{x}(\ket{0}-\ket{1}) &\xrightarrow{U_f} \sum_x^{2^n-1} (-1)^{f_s(x)}\ket{x} \\ 
                                        &=\sum_x^{2^n-1} (-1)^{s\cdot x} \ket{x} 
\end{align}
Then, we need to apply the Hadamard gate to this state. We will then write the following:
\begin{align}
  \sum_x^{2^n-1} (-1)^{s\cdot x} \ket{x} \xrightarrow[\text{\autoref{eq:hadamard_on_generic}}]{\mathcal{H}^{\otimes n}} \sum_x^{2^n-1} (-1)^{x\cdot s} 
  \sum_y^{2^n-1} (-1)^{x \cdot y} \ket{y} =\\
\sum_x^{2^n-1} \sum_y^{2^n-1} (-1)^{x\cdot s} (-1)^{x \cdot y} \ket{y} =\\
\sum_{x,y}^{2^n-1} (-1)^{(x\cdot s + x\cdot y)} \ket{y}
\end{align}
We now claim that the last expression $\sum_{x,y} (-1)^{x\cdot s + x\cdot y}\ket{y} = \ket{s}$. It is not straightforward to see so we can try to prove it (as
in \cite{noauthor_bernsteinvazirani_2022}).
We first can rewrite $x\cdot s + x\cdot y=x\cdot (s\oplus y)$. 
We can now take one fixed $\ket{y}$ and sum over the $\ket{x}$'s:
\begin{align}
  \sum_{x} (-1)^{x \cdot (s \oplus y)}
\end{align}
Now, if the chosen $\ket{y}$ is equal to the $s$, the term $s\oplus y$ will be clearly zero. Therefore, if the chosen $y$ 
happens to be the same as $s$, the probability of obtaining it will be maximum (we're adding all ones, as $(-1)^{x\cdot(s\oplus y)} = \sum (-1)^{0}$).
Consequently the only non-zero term is associated to the state $\ket{y}=\ket{s}$.

Thus, using this algorithm, we're able to find the state $\ket{y}=\ket{s}$ related to this "secret" string.
 kakaaa

\subsection{Simons algorithm}
Let's now consider the next algorithm, commonly known as the Simon's algorithm or the Simon's 
problem. As usual we will treat the concept mathematically, where it is nicely shown in Wikipedia \cite{noauthor_simons_2023}.

We first want to determine the problem statement. We are given some kind of function $f(x)$
Often, we define this problem through the concept of the whether this function is \textit{one-to-one} or 
\textit{two-to-one}. Both of these notions can be formally related to some function properties, namely surjection 
and injection. Anyways, the \textit{one-to-one} function means that for any input $x$ to the function $f(x)$, there's 
only one possible output. The two-to-one means that there's possibly several inputs for one single output. 
For example, $f(0)=0, f(1)=1, f(2)=2, f(3)=3$ - which is a one-to-one function. Now, an example of a two-to-one function is 
given by $f(0)=f(1)=0, f(2)=f(3)=1$. Note that for the inputs, they can be represented as binary strings, representing numbers.

One can define it in another way, which is fully equivalent. Indeed, the problem now goes as follows:
we're given a function $f(x)$ and the number/binary string $s$. We're now given a promise for $f(x)$:
given $f(x)$ and $s$, $f(x)=f(y)$ if and only if $x\oplus y \in \{0^{\otimes n}, s\}$. That is, the output 
will be same for two different inputs if and only if $x\oplus y$ is either $0^{\otimes n}$ or $s$. Here, we define 
the operation $x \oplus y$ to be the bitwise XOR. That is, $x\oplus y \equiv (x_1 \text{ XOR } y_1, x_2 \text{ XOR } y_2, ..., x_n \text{ XOR } y_n)$.

\begin{wraptable}{l}{7cm}
  \begin{tblr}{c}
    \hspace{1cm}
    \Qcircuit @C=0.75em @R=1em {
      \lstick{\ket{0}^{\otimes n}} & \qw & \gate{\mathcal{H}^{\otimes n}} & \qw & \multigate{1}{f} & \qw & \gate{\mathcal{H}^{\otimes n}} \qw &\qw \\
      \lstick{\ket{0}^{\otimes n}} & \qw & \qw                            & \qw & \ghost{f}         & \qw & \qw & \qw 
    }
  \end{tblr}
\end{wraptable}
\label{cirq:simons}

The goal of the algorithm is to determine the bitstring $s$. To schematically show that these two formulations (in terms of one-to-one \& two-to-one) 
and the last one are equivalent, we first note that $a \oplus b = 0^{\otimes n} \iff a=b$.
Another fact that we notice is the following: for some $x$ and $s$, we have that $x\oplus y = s$ is unique for $x$ if and only if $s \neq 0^{\otimes n}$. 
In other words, the output of the operation $s$ is uniquely determined by the inputs only if $s=0^{\otimes n}$. Therefore, 
we say that if $s\neq 0^{\otimes n}$, the function is two-to-one and if $s=0^{\otimes n}$, the function is one-to-one.

Let's now consider the circuit of the algorithm. The circuit is given in \autoref{cirq:simons}.
As usual, we're considering the initial state, which will go through the first $\mathcal{H}$ and we'll obtain the 
usual result 

\begin{align}
  \mathcal{H}^{\otimes n} \ket{0}^{\otimes n}=\frac{1}{\sqrt{2^n}}\sum_{x=0}^{2^n-1} \ket{x}
\end{align}

Then we will apply our quantum function evaluation to the state on both registers:

\begin{align}
  \sum_{x=0}^{2^n-1} \ket{x} \ket{0}^{\otimes n} \xrightarrow{f(x)} \sum_{x=0}^{2^n-1} \ket{x}\ket{f(x)\oplus 0} =\sum_{x=0}^{2^n-1} \ket{x}\ket{f(x)} 
\end{align}

Which is nothing but the state after the function evaluation and before the second $\mathcal{H}^{\otimes n}$ on the first register. 
We will then obtain as usual:
\begin{align}
  &\sum_{x=0}^{2^n-1} \ket{x}\ket{f(x)} \xrightarrow{(\mathcal{H}^{\otimes n})\otimes (\mathbbm{1}^{\otimes n}) } \\ 
  &\sum_{x=0}^{2^n-1} \biggl[ \sum_{y=0}^{2^n-1} (-1)^{x\cdot y}\ket{y} \biggr] \ket{f(x)}=\sum_{y=0}^{2^n-1}\ket{y} \biggl[ \sum_{x=0}^{2^n-1}(-1)^{x\cdot y}\ket{f(x)} \biggr]
\end{align}

which is our state that will be measured. In fact, as usual, we've simplified the multiplicative factor. Therefore, the "correct" state that we'll be 
measuring, will be as the one described above times the factor $\nicefrac{1}{(2^n)}$. Therefore for a certain $\ket{y}$, the probability of this 
happening will be given by 
\begin{align}
  \Bigg| \Bigg| \frac{1}{2^n} \sum_{x=0}^{2^n-1} (-1)^{x\cdot y} \ket{f(x)} \Bigg| \Bigg|^2
  \label{eq:simons_last_meas}
\end{align}
Now we consider the 2 possible cases (whether $f$ will be one-to-one or two-to-one). 

If the function $f(x)$ is one-to-one, the \autoref{eq:simons_last_meas} will give us the probability of measuring some ket $\ket{y}$.
An another way to write this probability in \autoref{eq:simons_last_meas}, we can write the probability for a state $\ket{y}$ to be measured will be 
given by $|\bra{y}\ket{x}|$. Therefore, we have that the probability is given by nothing but 
\begin{gather}
  \Bigg< \frac{1}{2^n} \sum_{x=0}^{2^n-1} (-1)^{x\cdot y}\ket{f(x)} \Bigg| \frac{1}{2^n} \sum_{x'=0}^{2^n-1}(-1)^{x' \cdot y} \ket{f(x')} \Biggr> \\
  (\frac{1}{2^n})^2 \sum_{\substack{x=0,x'=0\\ x,x'}} (-1)^{x\cdot y} (-1)^{x' \cdot y} \bra{f(x)}\ket{f(x')}, \text{ using that } \bra{a}\ket{a'}=\delta_{a,a'} \\
  (\frac{1}{2^n})^2 ((-1)^{0\cdot y} (-1)^{0\cdot y} \bra{f(0)}\ket{f(0)} + (-1)^{0\cdot y} (-1)^{1\cdot y} \bra{f(0)}\ket{f(1)} + ... + (-1)^{2^n-1\cdot y} (-1)^{2^n-2\cdot y} \bra{f(2^n-1)}\ket{f(2^n-2)} + (-1)^{2^n-1\cdot y} (-1)^{2^n-1\cdot y} \bra{f(2^n-1)}\ket{f(2^n-1))
\end{gather}
We know that if $\bra{f(x_i)}\ket{f(x_j)} == 1$ if $i == j$, therefore we know that $2^n$ of the terms will be zero. Therefore, we have that 
$(2^n-1)(2^N-1) - 2^n$ terms will be zero. Therefore, we have that the probability of measuring $\ket{y}$ will be given by
\begin{gather}
  We know that if $\bra{f(x_i)}\ket{f(x_j)} == 0$ if $i == j$ \\
   \frac{1}{2^{2n}} \sum_{x=0}^{2^n-1}(-1)^{2x\cdot y}= \frac{1}{2^{2n}} (2^n)\cdot(1)=\frac{1}{2^n}
   \label{eq:simons_inner_product}
\end{gather}
Which is not surprising, as since $f$ is one-to-one, we're going through all the basis vectors.

Now we consider the second case. That is, the case, where the function is not one-to-one. We can follow the nice trick shown 
in Wikipedia \cite{noauthor_simons_2023}. When the function is not one-to-one, this means therefore that for some inputs $x_1$ \& $x_2$,
we have the same outputs $f(x_1)=f(x_2)=z$, where $z$ is some kind of value in the range/domain of the function $f$. For the specific case, one may write for 
the probability for some chosen $\ket{y}$ as for \autoref{eq:simons_last_meas}.
\cite{noauthor_simons_2023}.
\begin{gather}
  \Bigg|\Bigg| \frac{1}{2^n}\sum_{x=0}^{2^n-1} (-1)^{x\cdot y} \ket{f(x)} \Bigg|\Bigg|^2 = 
  \Bigg|\Bigg| \frac{1}{2^n} \sum_{z\in \text{Range}(f)}((-1)^{y\cdot x_1} +(-1)^{y\cdot x_2})\ket{z} \Bigg| \Bigg|^2
  \label{eq:simons_z_range}
\end{gather}
\begin{wraptable}{l}{3cm}
  \centering
  \begin{tabular}{|c|c|c|}
    \hline
    $x_1$ & $x_2$ & $s$ \\
    \hline
    \hline
    0 & 0 & 0 \\
    \hline
    1 & 0 & 1 \\
    \hline
    0 & 1 & 1 \\
    \hline
    1 & 1 & 0 \\
    \hline
  \end{tabular}
  \caption{Truth table for XOR.}
  \vspace{-1.0cm}
  \label{table:xor_truth_table}
\end{wraptable}

Note that here we've changed the sum over $x$ i.e. over sum of the domain, to the sum over the range of its values $z$.
We now note another important property for the XOR operation. The truth table of XOR is given by the \autoref{table:xor_truth_table}.
From which we clearly see that $x_1 \otimes x_2 = s \iff x_1 \otimes s = x_2$. Therefore, if this is valid for one bit string, it 
will also be valid for any bit strings of any length, as the bitwise XOR only operates on pairs of bits.
Thus, by observing the property of $x_1 \otimes x_2 = s \iff x_1 \otimes s = x_2$, we can write the \autoref{eq:simons_z_range} 
as
\begin{gather}
  \Bigg|\Bigg| \frac{1}{2^n} \sum_{z\in \text{Range}(f)}((-1)^{y\cdot x_1} +(-1)^{y\cdot x_2})\ket{z} \Bigg| \Bigg|^2 =\\
  =\Bigg| \Bigg| \frac{1}{2^n} \sum_{z\in \text{Range}(f)}((-1)^{y\cdot x_1} +(-1)^{y\cdot (x_1 \otimes s)})\ket{z} \Bigg| \Bigg|^2 \\
  =\Bigg| \Bigg| \frac{1}{2^n} \sum_{z\in \text{Range}(f)}((-1)^{y\cdot x_1} +(-1)^{y\cdot x_1 \otimes y\cdot s)})\ket{z} \Bigg| \Bigg|^2 \\
  =\Bigg| \Bigg| \frac{1}{2^n} \sum_{z\in \text{Range}(f)} (-1)^{y\cdot x_1}(1+(-1)^{y\cdot s})\ket{z} \Bigg| \Bigg|^2 
\end{gather}
From where, we can observe some things: if the chosen $y$ happens to be such that $y\cdot s = 1$, then the probability (the sum given right above)
will be given by 0 (the factor in front of $\ket{z}$ will be 0). Now, if the value of the product $y\cdot s = 0$, it will not be zero.
instead, the sum becomes of the form 

\begin{gather}
  \Bigg| \Bigg| \frac{1}{2^n} \sum_{z\in \text{Range}(f)} (-1)^{y\cdot x_1}(2)\ket{z} \Bigg| \Bigg|^2 \\
  \Bigg| \Bigg| \frac{1}{2^{n-1}} \sum_{z\in \text{Range}(f)} (-1)^{y\cdot x_1}\ket{z} \Bigg| \Bigg|^2 
\end{gather}
Using the same trick as in \autoref{eq:simons_inner_product}, we can deduce that the probability will be given by $\frac{1}{2^{n-1}}$.

From these two cases, we can therefore deduce the following: the probability of measuring a certain $y$ if $s=0^{\otimes n}$ (the first case) will be given 
by $\frac{1}{2^n}$.
And the probability for a certain $y$ if $s\neq 0^{\otimes n}$, we have 2 (sub)cases. That is, depending on whether this $y$ will either obey
$y\cdot s=0$ or $y\cdot s\neq 0$. If it is the case of $y\cdot s=0$, we will have 0 probability of measuring this $\ket{y}$. If, on the other hand,
$y\cdot s\neq 0$, it will be given by $\frac{1}{2^{n-1}}$. 


Therefore, we deduce that in both cases, for some measured y, we will have that \underline{$y\cdot s =0$} (as in the first case, $s=0$ by definition and in the second, 
if it is not the case, the probability of measuring the $y$ state will be zero).

During the mathematical process, starting from \autoref{eq:simons_last_meas}, we considered some specific $\ket{y}$ that we'll get at the end/output.
So at the end we'll get several $\ket{y}$'s with probability, depending on different cases, as discussed above. The key point here is that 
these states, i.e. all $y$'s obey $y\cdot s = 0$.
As provided in \cite{noauthor_simons_2023}, we use the classical post processing in order to obtain the necessary $s$ as required in the problem statement.
It is important to note that, it may happen that there are "missing states", as there are sometimes 0 probability of observing some variables. 












\subsection{Fourier transform}
The quantum Fourier algorithm is, in my opinion, the "first useful algorithm" considered here...
Lets recall the (not very formal) definition of the classical Fourier transform. It is an operation $\mathcal{F}$ 
transforming some function $f(x)$ to an another function $f(y)$.
\begin{gather}
  f(x) \xrightarrow{\mathcal{F}} f(y) \; : \; f(y) = \int^{+\infty}_{-\infty}dx f(x) e^{-i2\pi y x}
\end{gather}
From that, one can define the concept of the discrete fourier transform. We use the analogy/mapping of the 
interval to a vector. That is, we can discretize some interval into a vector of $N$ elements $\vec{v}=(v_0, v_1, ..., v_{N-1})$.
We call the discretized version of the Fourier transform, without surprise, the Discrete Fourier Transform or DFT and defined by the mapping 
between two vectors $\vec{x}=(x_0, x_1,..., x_{N-1})$ to $\vec{y}=(y_0, y_1,...,y_{N-1})$.
\begin{gather}
  y_k = \sum_{n=0}^{N-1} x_n e^{-\frac{i2\pi}{N}kn}
\end{gather}

To a very similar manner, one can define the quantum fourier transform, which, instead of function-to-function and vector-to-vector mappings, 
there's quantum state to quantum state mapping. Namely, the Fourier transform maps some quantum state $\ket{x}\equiv \sum x_i \ket{i}$ to 
$\ket{y}=\sum y_i\ket{i}$ for some basis vectors $\{\ket{i}\}_{i\in \mathbb{N}}$. Thus we obtain the quantum Fourier transform's definition 
\begin{gather}
  \ket{y}\xrightarrow{\mathcal{F}} \ket{x}\\
  y_k=\frac{1}{\sqrt{N}} \sum_{n=0}^{N-1} x_n \omega_N^{kn} \text{ , with } \omega_N \equiv e^{i\frac{2\pi}{N}} \text{ and } k\in [0, N-1]
  \label{eq:qft_definition_components}
\end{gather}
Note that here the sign of $\omega$'s power is unimportant and is nothing but a convention. We also note the inverse Fourier transform, given by 
\begin{gather}
  \ket{x}\xrightarrow{\mathcal{F}} \ket{y}\\
  x_k=\frac{1}{\sqrt{N}} \sum_{n=0}^{N-1} y_n \omega_N^{-kn} \text{ , with } \omega_N \equiv e^{i\frac{2\pi}{N}} \text{ and } k\in [0, N-1]
\end{gather}
as shown in beautiful Wikipedia \cite{noauthor_quantum_2022}, we can represent the quantum Fourier transform with 
\begin{gather}
  \ket{x} \xmapsto{\mathcal{F}_{\text{quant}}} \frac{1}{\sqrt{N}} \sum_{k=0}^{N-1}\omega_N^{xk}\ket{k}
  \label{eq:qft_mapsto}
\end{gather}

Or, similarly, we can use the nice ket-bra representation of the (unitary) QFT operator \cite{noauthor_quantum_nodate}:
\begin{gather}
  \mathcal{F}=\frac{1}{\sqrt{N}} \sum_{x=0}^{N-1} \sum_{k=0}^{N-1} \omega_{N}^{xk}\ket{k}\bra{x}
\end{gather}

In addition to all that, we can also represent the quantum Fourier transform as a $(N-1)\times (N-1)$ matrix operation, acting on a vector.
\begin{gather}
  F_N=\frac{1}{\sqrt{N}} 
  \begin{bmatrix}
    1     & 1     & 1     & 1     & \cdots & 1    \\
    1     &\omega &\omega^2 & \omega^3 & \cdots & \omega^{N-1}\\
    1     &\omega^2 &\omega^4 & \omega^6 & \cdots & \omega^{2(N-1)}\\
    1     &\omega^3 &\omega^6 & \omega^9 & \cdots & \omega^{3(N-1)}\\
    \vdots &\vdots  & \vdots  & \vdots   & \ddots & \vdots        \\
    1      & \omega^{N-1}&\omega^{2(N-1)}&\omega^{3(N-1)} & \cdots & \omega^{(N-1)(N-1)}
  \end{bmatrix}
\end{gather}


Now, before considering some general cases of the Quantum Fourier transform and circuits, let's consider first some basic examples 
with small number of qubits and other considerations.

First of all \cite{noauthor_quantum_nodate}, we can consider the Fourier Transform as not changing the initial state, but rather 
changing it to a different representation in the so-called fourier basis. That is, $\mathcal{F}\ket{x}=\ket{\tilde{x}}$.


\paragraph{}
Now let's consider one qubit system given by the most general expression $\ket{\psi}=\alpha \ket{0} + \beta \ket{1}$.
The one qubit, as usual consists of 2 states (in the Z-basis). We can now look at all the previous expressions 
written above, to have that $N=2$. Using that, we can use, for example, the \autoref{eq:qft_mapsto}, or the component-by-component 
expression \autoref{eq:qft_definition_components}.
Thus, for the 1 qubit system, we have 2 components that we'll obtain. We will denote the components of the vector/state $\ket{y}$
by $y_0$ and $y_1$. The components of the "input vector" $\ket{\psi}$ with basis $\ket{0}$ and $\ket{1}$, are given by respectively $\alpha$
and $\beta$. Thus, using the \autoref{eq:qft_definition_components} and the definition of $\omega$ , we have by components:
\begin{gather}
  y_0= \frac{1}{\sqrt{N}}\sum_{n=0}^{N-1}\omega_{N}^{kn} x_n \stackrel{N=2, k=0}{=} 
  \frac{1}{\sqrt{2}}\biggl( \alpha e^{\frac{i2\pi}{2}\cdot0\cdot0} + \beta e^{\frac{i2\pi}{2}\cdot 0\cdot 1} \biggr)=
  \frac{1}{\sqrt{2}}(\alpha + \beta )\\
  y_1= \frac{1}{\sqrt{N}}\sum_{n=0}^{N-1}\omega_{N}^{kn} x_n \stackrel{N=2,k=1}{=} 
  \frac{1}{\sqrt{2}}\biggl( \alpha e^{\frac{i2\pi}{2}\cdot1 \cdot0} + \beta e^{\frac{i2\pi}{2}\cdot 1\cdot 1} \biggr)=
  \frac{1}{\sqrt{2}}(\alpha + \beta e^{i\pi})=\frac{1}{\sqrt{2}}(\alpha - \beta) 
\end{gather}

With that, we see that the state $\alpha\ket{0} + \beta\ket{1}$ gets transformed to 
$\frac{(\alpha+\beta)}{\sqrt{2}}\ket{0} + \frac{(\alpha - \beta)}{\sqrt{2}}\ket{1}$.
One may notice that this can be rewritten as 
$\frac{\alpha}{\sqrt{2}}(\ket{0}+\ket{1}) + \frac{\beta}{\sqrt{2}}(\ket{0}-\ket{1})$, which 
is nothing but $\alpha \ket{+} + \beta \ket{-}$. In other words, 
$\alpha \ket{0} + \beta \ket{1} \xrightarrow{\mathcal{F}_{N=2}} \alpha \ket{+} + \beta \ket{-}$, which 
is nothing but our Hadamard gate $\mathcal{H}$. Thus, the Fourier transform of 2 state qubit is 
the $\mathcal{H}$. As it is mentioned in \cite{noauthor_quantum_nodate}, we can write the obtained 
$\alpha \ket{0} + \beta \ket{1} \xrightarrow{\mathcal{F}_{N=2}} \tilde{\alpha} \ket{0} + \tilde{\beta} \ket{1}$
, which we call the \textit{Fourier basis}.

Now, still following the path given in \cite{noauthor_quantum_nodate}, we can try to describe 
this transformation for a generic $n$ qubit case. In this case, $N=2^n$. Then, we can write 
for the generic case:
\begin{gather}
  \mathcal{F}_N \ket{x} = \frac{1}{\sqrt{N}}\sum_{y=0}^{N-1} \omega_N^{yx}\ket{y} =\\ 
                        = \frac{1}{\sqrt{N}} \sum_{y=0}^{N-1} e^{i\frac{2\pi xy}{2^n}} \ket{y}
\label{eq:qft_generic_1}
\end{gather}
, now, we can use the fact that $\ket{y}$ represents a tensor product $\bigotimes_i\ket{y_i}$ , 
with $\ket{y_i}\in \{0,1\}$. We elaborate then further on the \autoref{eq:qft_generic_1}:
\begin{gather}
  \mathcal{F}_N \ket{x} = \frac{1}{\sqrt{N}} \sum_{y=0}^{N-1} e^{i2\pi (\sum_k \frac{y_k}{2^k})x} \ket{y_0 y_1 y_2...y_n}
  \label{eq:qft_generic_2}
\end{gather}
We note however that in \autoref{eq:qft_generic_2}, we used the fact that 
$\nicefrac{y}{2^n}=\sum_k^n \nicefrac{y_k}{2^k}$. This is true due to the definition of a binary 
number notation. Indeed, some number $s$ can be represented in binary in the form of 
$s=\sum_{k=1}^{n} s_k 2^k$, with $n$ being the most signicative bit. Thus, by dividing the whole expression 
by $2^n$, we will get $\nicefrac{s}{2^n}=\sum_{k=1}^{n} s_k 2^{k-n}$. The desired expression expression 
is obtained using the difference in the position of the most/least significant bits, here, 
followed in \cite{noauthor_quantum_nodate}\footnote{It is possible to check that this is true for 
any representations of binary numbers not depending on LSB/MSB's position}.
Now, continuing to expand \autoref{eq:qft_generic_2}, we have:
\begin{gather}
  \mathcal{F}_N \ket{x} = \frac{1}{\sqrt{N}} \sum_{y=0}^{N-1} e^{i2\pi (\sum_k \frac{y_k}{2^k})x}\ket{y_0 y_1 y_2...y_n} \\
  = \frac{1}{\sqrt{N}} \sum_{y=0}^{N-1} \prod_{k}^{n} e^{i2\pi \frac{y_k}{2^k}}\ket{y_0 y_1 y_2...y_n}
\end{gather}
Now, one can write the sum $\sum_{y=0}^{N-1}$ over the $\ket{y_0y_1y_2...y_k}$ 
as the sum over the components $y_k$ of the $\ket{y_0y_1y_2...y_k}$ as 
$\sum_{y_0=0}^{1}\sum_{y_1=0}\sum_{y_2=0}...\sum_{y_n=0}$.
The last step \cite{hosgood_introduction_nodate} is to simply rewrite as a 
product of different states. That is,
\begin{gather}
  \mathcal{F}_N \ket{x} = \frac{1}{\sqrt{N}} \sum_{y=0}^{N-1} \prod_{k}^{n} e^{i2\pi \frac{y_k}{2^k}}\ket{y_0 y_1 y_2...y_n} = 
  \bigotimes_{k=1}^{n}\biggl( \ket{0}+e^{2i\pi\frac{x}{2^k}}\ket{1} \biggr)
\end{gather}

\begin{table}[!hbt]
  \centering
  \begin{tabular}{|c|c|}
    \hline
    $CROT_k=
    \begin{bmatrix}
      \mathbbm{1}& 0 \\
      0 & UROT_k\\
    \end{bmatrix}
    $ & 
    $UROT_k = 
    \begin{bmatrix}
      1 & 0 \\
      0 & e^{\frac{2\pi i}{2^k}}\\
    \end{bmatrix}
    $\\
    \hline
  \end{tabular}
  \caption{The definitions of the $CROT_k$ and $UROT_k$ operators.}
  \label{table:qft_crot_urot}
\end{table}

Now, we're in position to consider the Quantum circuit of the fourier transform. Before that, we'll first consider the 
building blocks of the QFT, that is, the 2 gates-the $CROT_k$ and the $UROT_k$. Their representations are given in \autoref{table:qft_crot_urot}.
The action of the $CROT_k$ operator can also be described as follows \cite{noauthor_quantum_nodate}: 
\begin{gather}
  CROT_k \ket{0\psi} = \ket{0 \psi}\\
  CROT_k \ket{1\psi} = e^{\frac{2\pi i}{2^k}\psi}\ket{1 \psi}, \text{ with } \ket{\psi} \text{ is a binary string}
\end{gather}
We can now consider the circuit of the QFT.
First, for 2 qubits, the 2-qubit QFT:
\begin{table}[!hbt]
  \centering
\begin{tblr}{c}
  \Qcircuit @C=1em @R=1em{
    \lstick{\ket{x_1}} & \qw & \gate{\mathcal{H}} &  \ustick{A} \qw & \gate{UROT_2} & \qw \\
    \lstick{\ket{x_2}} & \qw & \qw                &  \qw            & \ctrl{-1}     & \qw
  }
\end{tblr}
\end{table}
Let's now write our usual description of the circuit, starting at some states $\ket{x_1}\otimes\ket{x_2}$. 
The first gate will give the state at $A$: $\ket{\psi_A}=\mathcal{H}\otimes \mathbbm{1} (\ket{x_1}\otimes \ket{x_2}) = 
\frac{1}{\sqrt{2}}(\ket{0}+e^{\frac{2\pi i}{2}x_1}\ket{1})\otimes \ket{x_2}$. Now, we can apply the $CROT_2$ operator, which, in bra-ket notation gives 
$CROT_2=\mathbbm{1}\otimes\ket{0}\bra{0}+UROT_2\otimes \ket{1}\bra{1}$. Thus, by applying this operator to the state at A,
we get 
\begin{gather}
CROT_2\biggl(\frac{1}{\sqrt{2}} \bigl(\ket{0}+e^{\frac{2\pi i}{2}x_1}\ket{1}\bigr)\otimes \ket{x_2}\biggr)=\\
=\biggl(\mathbbm{1}\otimes\ket{0}\bra{0}+UROT_2\otimes \ket{1}\bra{1}\biggr) \biggl(\frac{1}{\sqrt{2}}(\ket{0}\otimes\ket{x_2}+e^{\frac{2\pi i}{2}x_1}\ket{1}\otimes\ket{x_2})\biggr)=\\
=\frac{1}{\sqrt{2}}\biggl(\ket{0}\otimes \bra{0}\ket{x_2} + e^{\frac{2\pi i}{2}x_1}\ket{1}\otimes \bra{0}\ket{x_2}) \biggr)+\\
+ \frac{1}{\sqrt{2}}\biggl(\ket{0}\otimes \bra{1}\ket{x_2} + 
  [e^{\frac{2\pi i}{2}x_2}+e^{\frac{2\pi i}{2^2}x_2}]\otimes\bra{1}\ket{x_2}\biggr)
\end{gather}
which will give different outcomees for different inputs.
Equivalently, in a less messy manner, \cite{noauthor_quantum_nodate}, we can write:
\begin{gather}
  CROT_2\biggl(\frac{1}{\sqrt{2}} \bigl(\ket{0}+e^{\frac{2\pi i}{2}x_1}\ket{1}\bigr)\otimes \ket{x_2}\biggr)\stackrel{\text{ctrl.-2, targ.-1}}{=}\\ 
  = \frac{1}{\sqrt{2}}\biggl( \ket{0}+ [e^{\frac{2\pi i}{2}x_1} + e^{\frac{2\pi i}{2^2}x_2}]\ket{1}\biggr) \otimes \ket{x_2}
\end{gather}
With this idea, let's finally try to generalize it for a more generic input of $n$ qubits.
This by \textit{replacing} the $\otimes\ket{x_2}$ by $\otimes \ket{x_2x_3...x_n}$.
Therefore, the state after the first $UROT_2$ will be given by 
$\frac{1}{\sqrt{2}}\biggl( \ket{0}+ [e^{\frac{2\pi i}{2}x_1} + e^{\frac{2\pi i}{2^2}x_2}]\ket{1}\biggr) \otimes \ket{x_2x_3...x_n}$
Then, the next step will undergo the next $n$ controlled $UROT_k$ operators with the $k$ qubit being the controlled qubit and 
the first qubit being the target. After all the $n$ gates (using the same logic as for the 2-qubit systems), we will get the 
state \cite{noauthor_quantum_nodate}:
\begin{gather}
\frac{1}{\sqrt{2}}\biggl( \ket{0}+ [e^{\frac{2\pi i}{2}x_1} + 
e^{\frac{2\pi i}{2^2}x_2}\;...+...\; e^{\frac{2\pi i}{2^{n-1}}x_{n-1}} + e^{\frac{2\pi i}{2^{n}}x_{n}}]\ket{1}\biggr) \otimes \ket{x_2x_3...x_n}
\end{gather}
Now, we can notice the already mentioned \textit{trick} in \autoref{eq:qft_generic_2}. That is, the fact that 
in binary representation, $x=2^{n-1}x_1+2^{n-2}x_2 + ...+ 2^1x_{n-1} + 2^0x_n$. Thus, the sum expression in the square 
brackets (factor of $\nicefrac{x_j}{2^j}$) is nothing but $\nicefrac{x}{2^n}$, giving us the state 
\begin{gather}
  \frac{1}{\sqrt{2}}\biggl( \ket{0}+e^{\frac{2\pi i}{2^n}x}\ket{1}\biggr) \otimes \ket{x_2x_3...x_n}
  \label{eq:qft_2nd_generic_step}
\end{gather}

So up to now we performed the following things on the 1-n qubits: Apply the $\mathcal{H}$ followed by $n$ $UROT_k$ operators with 
the first qubit as the target \textbf{and} the $[\![1,n]\!]$ \footnote{The notation $[\![1,n]\!]$ means a discrete 
interval. For example, $[\![2,5]\!] \equiv 2,3,4,5$} being the control ones. 

\vspace{-0.2cm}
The next steps are exactly the same, but now, the 2nd qubit becoms the target and the $[\![2,n]\!]$ become the controls.
This is performed then for $[\![3,n]\!]$, $[\![4,n]\!]$, etc... It can be easily shown from the \autoref{eq:qft_2nd_generic_step}, that, by applying subsequent 
operations described above, we will subsequently get 
\vspace{-0.4cm}
\begin{gather}
\frac{1}{\sqrt{2}}\biggl( \ket{0}+e^{\frac{2\pi i}{2^n}x}\ket{1}\biggr)\otimes \ket{x_2x_3...x_n} \rightarrow
\label{eq:qft_first_n_steps}
\end{gather}
\vspace{-0.75cm}
\begin{gather}
\frac{1}{\sqrt{2}}\biggl( \ket{0}+e^{\frac{2\pi i}{2^n}x}\ket{1}\biggr)\otimes \frac{1}{\sqrt{2}}\biggl( \ket{0}+e^{\frac{2\pi i}{2^{n-1}}x}\ket{1}\biggr)\otimes \ket{x_3...x_n} \rightarrow ...
\label{eq:qft_second_n_steps}
\end{gather}
\vspace{-0.75cm}
\begin{gather}
\frac{1}{\sqrt{2}}\biggl( \ket{0}+e^{\frac{2\pi i}{2^n}x}\ket{1}\biggr)\otimes \frac{1}{\sqrt{2}}\biggl( \ket{0}+e^{\frac{2\pi i}{2^{n-1}}x}\ket{1}\biggr) \otimes ...\otimes \frac{1}{\sqrt{2}}\biggl( \ket{0}+e^{\frac{2\pi i}{2^0}x}\ket{1}\biggr)
\label{eq:qft_last_n_steps}
\end{gather}
\vspace{-0.4cm}

Let's finally try to implement and comment the QFT circuit and make brief comments on them (I'll write $U_k$ instead of $UROT_k$ for space saving's reasons)

\begin{table}[!hbt]
  \centering
\begin{tblr}{c}
\Qcircuit @C=0.5em @R=0.6em {
  \lstick{\ket{x_1}}& \qw  & \gate{\mathcal{H}}& \gate{U_2}& \gate{U_3}   &\qw&&& \gate{U_n}      &\qw             &\qw                 & \qw       &\qw       &&\hdots&& &\qw       &&\hdots&\hdots&\hdots&&&\qw & \qw& \qw &\qw& \qw& \qw \\
  \lstick{\ket{x_2}}& \qw  & \qw               & \ctrl{-1}    & \qw       &\hdots&&& \qw          &\qw             &\gate{\mathcal{H}}  & \gate{U_2}&\gate{U_3}&&\hdots&& &\gate{U_n}&&\hdots&\hdots&\hdots&&&\qw & \qw& \qw &\qw& \qw& \qw \\
  \lstick{\ket{x_3}}& \qw  & \qw               & \qw          & \ctrl{-2} &\hdots&&& \qw          &\qw             &\qw                 & \ctrl{-1} &\qw       &&\hdots&& &\qw       &&\hdots&\hdots&\hdots&&&\qw & \qw& \qw &\qw& \qw& \qw \\
  \lstick{\ket{x_4}}& \qw  & \qw               & \qw          & \qw       &      &&& \qw          &\qw             &\qw                 & \qw       &\ctrl{-2} &&\hdots&& &\qw       &&\hdots&\hdots&\hdots&&&\qw & \qw& \qw &\qw& \qw& \qw \\
  & & \qvdots \\
  \lstick{\ket{x_{n-1}}}&\qw& \qw              & \qw          &\qw        & \hdots  &&& \qw       &\qw             &\qw                 & \qw       &\qw       &&\hdots&& &\qw       &&\hdots&\hdots&\hdots&&&\qw & \gate{\mathcal{H}}& \gate{U_2} &\qw & \qw & \qw\\
  \lstick{\ket{x_n}}&\qw   & \qw               & \qw          & \qw       & \hdots  &&& \ctrl{-6} &\qw             &\qw                 & \qw       &\qw       &&\hdots&& &\ctrl{-5} &&\hdots&\hdots&\hdots&&&\qw &\qw                & \ctrl{-1}  &\qw & \gate{\mathcal{H}} &\qw 
}
\end{tblr}
\caption{The QFT algorithm, where successive $UROT_k$ operations are performed. The circuit in fact is composed of 2 types of building blocks. We first chose the target qubit and then 
apply the $UROT_k$ operation with $k$ being the control qubit of those successive operations. This operation (where the 1 qubit is the target is 
described in the \autoref{eq:qft_first_n_steps}). Then, we proceed with the second qubit to be the target and we perform the same operation now for the $[\![2,n]\!]$ qubits as control.
(as per \autoref{eq:qft_second_n_steps}). We proceed then for the rest of the qubits as targets, to finally arrive at the last qubit $n-1$ and $n$. The result of these operations 
is given in \autoref{eq:qft_last_n_steps}.}
\label{cirq:qft}
\end{table}











\subsection{Phase estimation}
\label{subsec:phase_estim}

The phase estimation is an important algorithm, that does what its name suggests - to 
estimate the phase of a system. That it, suppose there exists a state $\ket{\psi}$. Suppose then 
there exists an operator $U\ket{\psi} = e^{2i\pi\theta}\ket{\psi}$. The goal of the algorithm 
is thus to find the eigenvalue's $e^{2i\pi\phi\theta}$ argument $\theta$. The input of the algorithm is therefore 
the state $\ket{\psi}$, as well as an input vector $\ket{0}^{\otimes ^m}$, which server as the counting 
vector. So this $\ket{0}^{\otimes ^m}$ will accumulate the counting of the phase $\theta$.

Let's consider the general circuit of the phase estimation algorithm \cite{noauthor_quantum_phase_estim}
\cite{noauthor_quantum_phase_estim_wiki} and try to mathematically analyze the circuit in a more 
mathematical way.

\begin{table}[ht!]
  \centering
  \begin{tblr}{c}
    \Qcircuit @C=1em @R=1em{
      \lstick{\ket{0}}    & \qw & \gate{\mathcal{H}} & \ctrl{5}          & \qw & \qw               &\qw&\hdots&&\qw & \qw        & \qw                &\qw& \multigate{5}{\mathcal{F}^{-1}} & \qw \\
      \lstick{\ket{0}}    & \qw & \gate{\mathcal{H}} & \qw               & \qw & \ctrl{4}          &\qw&\hdots&&\qw & \qw        & \qw                &\qw& \ghost{\mathcal{F}^{-1}}        & \qw \\
      \hdots              & \qw & \hdots             &                   & \qw & \hdots            &   &\hdots&&\qw & \qw        & \qw                &\qw& \ghost{\mathcal{F}^{-1}}        & \qw \\
      \lstick{\ket{0}}    & \qw & \gate{\mathcal{H}} & \qw               & \qw & \qw               &\qw&\hdots&&\qw & \ctrl{2}   &  \qw               &\qw& \ghost{\mathcal{F}^{-1}}        & \qw \\
      \lstick{\ket{0}}    & \qw & \gate{\mathcal{H}} & \qw               & \qw & \qw               &\qw&\hdots&&\qw & \qw        & \ctrl{1}           &\qw& \ghost{\mathcal{F}^{-1}}        & \qw \\
      \lstick{\ket{\psi}} & \qw & \qw                &\gate{U^{2^{m-1}}} & \qw & \gate{U^{2^{m-2}}}&\qw&\hdots&&\qw & \gate{U^{2^1}} & \gate{U^{2^0}} &\qw& \ghost{\mathcal{F}^{-1}}        & \qw 
    }
  \end{tblr}
  \caption{}
  \label{cirq:phase_estim}
\end{table}

Note that there are some similarities with the QFT, that is, the operator $U$ raised to the powers $2^{k}$. 

So as usual, we have that the initial state is given by nothig but $\ket{0}^{\otimes^m}\otimes \ket{\psi}$. Then, the next step is to apply the $\mathcal{H}$ 
on the counting bits. We have then 
\begin{gather}
  \ket{\psi}^{\otimes m}\ket{\psi} \xrightarrow{\mathcal{H}^{\otimes^m}\otimes \mathbbm{1}} \frac{1}{2^{\nicefrac{2}{n}}} \bigl (\ket{0} + \ket{1} \bigr)^{\otimes^{m}}\otimes \ket{\psi}
\end{gather}

Then, we sequencially apply the different $U^{2^k}$ operators. This is similar to the operators found in the QFT. As the eigenvalue of the $U$ operator 
is given by $e^{2\pi i \theta}$ the power of this operator can be expressed by: 
\begin{gather}
  U^{2^k}\ket{\psi} = e^{2\pi i 2^{k}\theta}\ket{\psi}
\end{gather}

Therefore, the consecutive applications of these $U^{k}$ are given below. We should also remember, that the controlled $U$ operator in the case of e.g. 2 qubits with 
the 0th as the control and the 1st as the target, will be given by $U = \ket{0}\bra{0}\otimes \mathbbm{1} + \ket{1}\ket{1}\otimes U$
\begin{gather}
  \frac{1}{2^{\frac{n}{2}}} \bigl (\ket{0} + \ket{1} \bigr)^{\otimes^{m}}\otimes \ket{\psi} \xrightarrow{\text{apply the controlled }U^{2^{m-1}}} \\
  \frac{1}{2^{\frac{n}{2}}}\biggl(\ket{0}\bra{0}\otimes (\mathbbm{1}^{\otimes^{m-1}}) \otimes \mathbbm{1} + \ket{1}\bra{1}\otimes (\mathbbm{1}^{\otimes^{m-1}}) \otimes U \biggr) 
  \frac{1}{2^{\frac{n}{2}}}\biggl[\frac{1}{2^{\frac{2}{n}}}\bigl (\ket{0} + \ket{1} \bigr)\otimes \bigl (\ket{0} + \ket{1} \bigr)^{\otimes^{m-1}}\otimes \ket{\psi} \biggr]= \\
  = \frac{1}{2^{\frac{n}{2}}}\biggl[ \ket{0}\otimes (\ket{0}+\ket{1})^{\otimes^{m-1}} \otimes \ket{\psi} + \ket{1}\otimes (\ket{0}+\ket{1})^{\otimes^{m-1}}\otimes U\ket{\psi} \biggr] = \\
  \frac{1}{2^{\frac{n}{2}}}\biggl[ \ket{0}\otimes (\ket{0}+\ket{1})^{\otimes^{m-1}} \otimes \ket{\psi} + \ket{1}\otimes (\ket{0}+\ket{1})^{\otimes^{m-1}}\otimes e^{2\pi i2^{m-1}\theta}\ket{\psi} \biggr] = \\
  = \frac{1}{2^{\frac{n}{2}}}\biggl[ \ket{0} + e^{2\pi i2^{m-1}\theta}\ket{1} \biggr] \otimes \biggl[ \ket{0}+\ket{1} \biggr]^{\otimes^{m-1}}\otimes \ket{\psi}
\end{gather}
This application can be performed several times. At the end, using the similar development, it is possible to 
show that after the consecutive applications of the $U$ operator, we obtain at the end 
\begin{gather}
  \frac{1}{2^{\frac{2}{n}}} \biggl[ \ket{0} + e^{2\pi i2^{m-1}\theta}\ket{1}  \biggr] \otimes \biggl[  \ket{0} + e^{2\pi i2^{m-2}\theta}\ket{1} \biggr] \otimes 
  \cdots \otimes \biggl[  \ket{0} + e^{2\pi i2^{0}\theta}\ket{1} \biggr] \otimes \ket{\psi} = \\
  = \frac{1}{2^{\frac{2}{n}}}\biggl[ \sum_{k=0}^{2^n-1} e^{2\pi i k \theta} \ket{k} \biggr]\otimes \ket{\psi}
\end{gather}

So this is the state before we apply the inverse Fourier transform. However, let's first quickly recall how does the simple Fourier transform work.
\begin{gather}
  \ket{y} \mapsto \biggl(\ket{0} + e^{\frac{2\pi i}{2^1}x}\ket{1}\biggr)\otimes \biggl(\ket{0} + e^{\frac{2\pi i}{2^2}x}\ket{1}\biggr) \cdots \biggl(\ket{0} + e^{\frac{2\pi i}{2^{n}}x}\ket{1}\biggr) = \\
  \ket{y} \mapsto \biggl(\sum_{k=0}^{N-1}\omega_N^{yk}\ket{k} \biggr)
\end{gather}
The inverse fourier transform simply swaps the sign of the exponential of the $\omega_N$.
That is, we can define the operator of the inverse Fourier transform: 
\begin{gather}
  \ket{y} \mapsto \biggl(\sum_{k=0}^{N-1}\omega_N^{-yk}\ket{k} \biggr) 
\end{gather}
and the operator: 
\begin{gather}
  \widehat{\mathcal{F}^{-1}} = \frac{1}{2^\frac{n}{2}}\sum_{x=0}^{2^n-1}\sum_{k=0} \omega_N^{-xk}\ket{k}\bra{x} \\
  \widehat{\mathcal{F}^{-1}} = \frac{1}{2^\frac{n}{2}}\sum_{x=0}^{2^n-1}\sum_{k=0} e^{-\frac{2\pi ixk}{2^n}}\ket{k}\bra{x}
\end{gather}

So we have that the state that the $\mathcal{F}^{-1}$ must be applied to is given by 

\begin{gather}
  \frac{1}{2^{\frac{n}{2}}}\biggl[ \sum_{k=0}^{2^n-1} e^{2\pi i k \theta} \ket{k} \biggr]\otimes \ket{\psi} \equiv 
  \frac{1}{2^{\frac{n}{2}}}\biggl[ \sum_{y=0}^{2^n-1} e^{2\pi i y \theta} \ket{y} \biggr] 
\end{gather}

Let's now try to apply the inverse Fourier in it: 

\begin{equation}
\begin{split}
  \widehat{\mathcal{F}^{-1}} = \frac{1}{2^{\frac{n}{2}}} \biggl[ \sum_{y=0}^{2^n-1} e^{2\pi i y \theta} \ket{y} \biggr]  = \\
  \frac{1}{2^n}\sum_{x=0}^{2^n-1}\sum_{k=0} e^{-\frac{2\pi ixk}{2^n}}\ket{k}\bra{x} \biggl[ \sum_{y=0}^{2^n-1} e^{2\pi i y \theta} \ket{y} \biggr] = \\
  \frac{1}{2^n} \sum_{y=0}^{2^n-1} \sum_{x=0}^{2^n-1}\sum_{k=0}^{2^n-1} e^{-\frac{2\pi ixk}{2^n}} e^{2\pi i y \theta} \ket{k}\bra{x} \ket{y} \stackrel{\bra{x}\ket{y}=\delta_{xy}}= \\
  \frac{1}{2^n} \sum_{y=0}^{2^n-1} \sum_{x=0}^{2^n-1}\sum_{k=0}^{2^n-1} e^{-\frac{2\pi ixk}{2^n}} e^{2\pi i y \theta} \ket{k}\bra{x} \ket{y} \stackrel{\bra{x}\ket{y}=\delta_{xy}}= \\
  \frac{1}{2^n} \sum_{y=0}^{2^n-1}\sum_{k=0}^{2^n-1} e^{-\frac{2\pi iyk}{2^n}} e^{2\pi i y \theta} \ket{k} = 
  \frac{1}{2^n} \sum_{y=0}^{2^n-1}\sum_{k=0}^{2^n-1} e^{-2\pi iy(\frac{k}{2^n}- \theta)} \ket{k} = \\
  = \frac{1}{2^n} \sum_{y=0}^{2^n-1}\sum_{k=0}^{2^n-1} e^{\frac{-2\pi iy}{2^n}(k - \theta 2^n)} \ket{k}
\end{split}
\end{equation}

Which is the last state after the inverse Fourier transofrm. To be more precise, 
\begin{gather}
  \frac{1}{2^n} \sum_{y=0}^{2^n-1}\sum_{k=0}^{2^n-1} e^{\frac{-2\pi iy}{2^n}(k - \theta 2^n)} \ket{k} \otimes \ket{\psi}
\end{gather}
Now, we can ask ourselves, how are distributed the probabilities of measuring this final state? 
It is clear, that if $k=\theta 2^n$, the probability of measuring the state $\ket{k} = \ket{\theta 2^{n}}$ is the maximum. Indeed, in this case, 
the exponential, $e^{\theta 2^n - (k=)\theta 2^n}$ will have the maximum value. 

Therefore, at the end, we will know what is theta, by measuring the first m qubits.



















\newpage
\bibliographystyle{plainurl}
\printbibliography



\end{document}
