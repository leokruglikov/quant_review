\section{Shor's algorithm}
The Shor's algorithm is one of the most famous algorithms in quantum computing.
It has the potential ability to factor integers with polynomial time. The process of integer factoring 
is very important in computer security, that is, in cryptography. We will however not go into 
great details into the deep mathematics behind cryptography, but only what it is necessary for 
the understanding of the algorithm.

To consider the Shor's algorihtm, we define the function that is working with natural numbers
and with modular arithmetic \footnote{Without going into detail of modular arithmetic, let's give 
some examples, that will help us to understand it. $44 \stackrel{\text{mod }7}{\equiv} 2$, as $44=6\cdot7 + 2$. 
$34 \stackrel{\text{ mod }9} = 7$, as $34 = 3\cdot 9 + 7$. So this is the number that must be added to the maximum 
number of times that the first number will go into second (e.g. 7 or 9 will go into 44 or 34 respectively).}. 
We define therefore 
the function 
\begin{gather}
  f(x) = a^{x} \text{ mod } N \\
  f(x) \stackrel{\text{ mod } N}{\equiv} a^x
\end{gather}

We note that this function is clearly periodic. 
In addition to that, one also defines the notion of \textsl{period} for such function, denoting 
$t$ or $r$ \cite{noauthor_shors_nodate}, which is the smallest number $t$ such that 
\begin{gather}
  f(x=t) = a^t \text{ mod } N = 1 \\
  f(x=t) = 1 \stackrel{\text{mod } N} = a^t
\end{gather}


Now, we get to the quantum mechanical description of this algorithm. One defines as usual the (unitary) operator $U$:
\begin{gather}
  U\ket{y} = \ket{ay \text{ mod } N}
\end{gather}

Let's check how this operator acts sequentially for the example of $a=5$ and N=28...

\begin{equation}
\begin{split}
  U\ket{1}=\ket{5} \\
  U^2\ket{1} = U\ket{5} = \ket{5\cdot 5 \text{ mod } 28} = \ket{3} \\
  U^3\ket{1} = U\ket{3} = \ket{5\cdot 3 \text{ mod } 28} = \ket{13} \\
  U^4\ket{1} = U\ket{13} = \ket{5\cdot 13 \text{ mod } 28} = \ket{9} \\
  U^5\ket{1} = U\ket{9} = \ket{5\cdot 9 \text{ mod } 28} = \ket{17} \\
  U^6\ket{1} = U\ket{17} = \ket{5\cdot 17 \text{ mod } 28} = \ket{1} 
\end{split}
\end{equation}

We thus again came back to the state $\ket{1}$. This means that for the case of 
these numbers (i.e. $a=5$ and $N=28$), the period of the function is $6$. That is, 
$5^6 \stackrel{\text{ mod } 28}{\equiv} 1$.

In a similar fashion, one can define a vector, having an interesting property. 
Namely, being the eigenvector of this operator. The eigenvector 
is in fact the superposition of the aformentioned vectors. 
The superposition is given by 
\begin{gather}
  \ket{u} = \sum_{k=0}^{r-1} \ket{a^k\text{ mod }N}
\end{gather}

In the specific case of $a=5$ and $N=28$, we have that 
\begin{gather}
  \ket{u} = \frac{1}{4}\biggl(\ket{1} + \ket{13} + \ket{9} + \ket{17} \biggr)
\end{gather}

It is indeed an eigenket:
\begin{equation}
\begin{split}
  U\ket{u} = U\frac{1}{4}\biggl(\ket{1} + \ket{13} + \ket{9} + \ket{17}\biggl) = \\
    \frac{1}{4} \biggl(U\ket{1} + U\ket{13} + U\ket{9} + U\ket{17}\biggl) = \\
      \frac{1}{4} \biggl(\ket{13} + \ket{9} + \ket{17} + \ket{1}\biggl) = \\
        \frac{1}{4}\biggl(\ket{1} + \ket{13} + \ket{9} + \ket{17}\biggl) = \ket{u}
\end{split}
\end{equation}

We can show it for a general case for the $\ket{u} = \sum_{k=0}^{r-1} \ket{a^k \text{ mod } N}$. 
The main property of some state $\ket{a^{r-1} \text{ mod }}$ is 
$U\ket{a^{r-1} \text{ mod }} = \ket{a^{r} \text{ mod }} = \ket{1}$.

So we show that this is indeed the eigenket of the operator: 
\begin{equation}
  \begin{split}
    U\ket{u} = U\sum_{k=0}^{r-1} \ket{a^k \text{ mod } N} = \sum_{k=0}^{r-1} \ket{a^{k+1} \text{ mod } N} = \\
    = \sum_{k'=1}^{r-1} \ket{a^k \text{ mod } N} + \ket{a^{r = (k+1)} \text{ mod } N} = \\ 
    \sum_{k'=1}^{r-1} \ket{a^k \text{ mod } N} + 
    \ket{a^{0(=k)} \text{ mod } N} = \sum_{k=0}^{r-1} \ket{a^k \text{ mod }N} 
  \end{split}
  \label{eq:shors_eigenvec_proof_1}
\end{equation}

We can now consider another state: 
\begin{gather}
  \ket{u_1} = \sum_{k=0}^{r-1}e^{-\frac{2\pi ik}{r}} \ket{a^k \text{ mod } N}
  \label{eq:shors_eigenvec}
\end{gather}

In fact, it is also an eigenstate of the operator $U$.
One can show that the vector in \autoref{eq:shors_eigenvec} is also an eigenket of the operator $U$. 

\begin{equation}
  \begin{split}
    U \sum_{k=0}^{r-1} e^{-\frac{2\pi ik}{r}} \ket{a^k\text{ mod } N} = \sum_{k=0}^{r-1} e^{-\frac{2\pi ik}{r}} \ket{a^{k+1}\text{ mod } N} = \\
    = e^{\frac{2\pi i}{r}} \sum_{k=0}^{r-1} e^{-\frac{2\pi i(k+1)}{r}} \ket{a^{k+1}\text{ mod }N}
  \end{split}
\end{equation}
We now do the similar trick as for \autoref{eq:shors_eigenvec_proof_1}. To obtain 
\begin{equation}
  \begin{split}
    e^{\frac{2\pi i}{r}} \sum_{k=0}^{r-1} e^{-\frac{2\pi i(k+1)}{r}} \ket{a^{k+1}\text{ mod }N} = 
    e^{\frac{2\pi i}{r}} \sum_{k=0}^{r-1} e^{-\frac{2\pi ik}{r}} \ket{a^k\text{ mod } N} \\
  \end{split}
\end{equation}

In fact, in the general case, one can write the following:
\begin{gather}
  U \sum_{k=0}^{r-1}e^{-\frac{2\pi i sk}{r}}\ket{a^k \text{ mod }N} \equiv U\ket{u_s} = e^{\frac{2\pi is}{r}}\ket{u_s}
\end{gather}
The idea here is that the state $\ket{u_s}$ is an eigenvector of the operator $U$ and its eigenvalue is given by $e^{\frac{2\pi is}{r}}$.
Therefore, one can remember the fact that in \autoref{subsec:phase_estim}, we were estimating the phase of the eigenvalue 
of the operator $U$. In this case, the phase $\theta$ is given by $\theta = \frac{s}{r}$. Therefore, one can estimate the phase, thus obtain 
the so-wanted period $r$.

















