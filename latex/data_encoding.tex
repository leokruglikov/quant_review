\section{Data encoding}
In order to use the quantum algorithms in question, 
one should \textit{feed} the data to the quantum circuit. This is where the data encoding comes into place. 
There are various ways to encode the data and let's consider some of them.

\subsection*{Basis encoding}
Basis encoding is the most straightforward way to 
encode data. That is, suppose we want to treat an input composed of 4 elements. That is, 
we can encode the 4 states by associating $1\mapsto \ket{0} = \ket{00}$, $2\mapsto \ket{1} = \ket{01}$, 
$3\mapsto \ket{2} = \ket{01}$ and $4\mapsto \ket{3} = \ket{11}$. The input in the geral case of 2 qubits will therefore be 
given by 
\begin{equation}
  \ket{\psi}_{\text{in}} = \alpha_{00}\ket{00} + \alpha_{01}\ket{01} + \alpha_{10}\ket{10} + \alpha_{11}\ket{11}
\end{equation}

We therefore set the corresponding coefficient $\alpha $ to one for the data we want to send to the quantum 
circuit. For example, if one wants to feed the quantum circuit the number $2$, then the input state will look like 
\begin{equation}
  \ket{\psi}_{\text{in}} = 0\ket{00} + 1\ket{01} + 0\ket{10} + 0\ket{11} = \ket{01}
\end{equation}

On sees that this method is not fully optimal for complex datasets. For example, representing floating point numbers. 
For one simple floating point number, using the convention of classical representation, one would need 32 bits, therefore 
32 qubits for representing a floating point number.


\subsection*{Amplitude encoding}
The next method is the amplitude encoding. This method is quite straightforward. 
Namely, suppose one wants to encode some kind of vector $\vec{x}$. The encoding is given by 
\begin{equation}
  \vec{x} = 
  \begin{pmatrix}
    x_1 \\
    x_2 \\
    \vdots \\
    x_N
  \end{pmatrix}
  \xrightarrow{} \ket{\psi_{\vec{x}}} = \alpha \sum_{j = 0}^{n-1} x_j \ket{j} \qquad \text{ , with } \alpha \text{- a normalization constant}
\end{equation}
In a similar manner, one may represent a matrix/tensor. For example, a matrix $\vector{A}$ can be defined as 
$\ket{\psi_{\vec{A}}} = \alpha \sum_i \sum_j a_{i,j}\ket{i}\ket{j}$, where $a_{i,j}$ represent the matrix elements $A_{i,j}$.
For example, one may encode the vector $x = (1.0, 2.0, 0.4, 8.0)$. The corresponding encoding will then be 
$\ket{\psi_{\vec{x}}} = 1.0\ket{00} + 2.0\ket{01} + 0.4\ket{10} + 8.0\ket{11}$.
The normalization constant will be given by $\alpha = |\bra{\psi_{\vec{x}}}\ket{\psi_{\vec{x}}}|^\nicefrac{1}{2}$.

\subsection*{Phase or time evolution encoding}
Before considering the encoding, one remembers the notion of time-evolution in quantum 
mechanics. Let $\ket{\psi_0}$ be the initial state. Using the Stone's theorem, one defines the time-evolution 
operator, commonly denoted as $U(t)$. For a time-independent Hamiltonian, it is given by 
$U(t)=e^{-it\widehat{H}}$, where $\widehat{H}$ is the Hamiltonian operator.

Now, we move away from the time-evolution operator (this was a simple analogy). That is, the time $t$ is simply a real number, and the 
operator $\widehat{H}$ can be any operator, for example, any of the Pauli operators.
Now, one once again remembers the concept from quantum mechanics about the representation of the pauli operators $\widehat{\sigma}_{\{x,y,z\}}$. 
The exponent of a Pauli operator is given by 
\begin{equation}
  e^{ia(\widehat{n}\cdot \vector{\widehat{\sigma}})} = \mathbb{1}\cos(a) + i(\widehat{n}\cdot \vector{\widehat{\sigma}})\sin(a)
  \label{eq:general_pauli}
\end{equation}

Therefore, one can take the Pauli operator $\widehat{\sigma}$. One can therefore 
encode any number $x$ using the the state $\ket{0}$. This state is used as $\widehat{\sigma}_y$ applied on 
states $\ket{1}$ and $\ket{0}$ give a multiplication factor $i$. Namely, 
$\widehat{\sigma}_y \ket{0} = i\ket{1}$. One can therefore get rid of the $i$ in \autoref{eq:general_pauli}.
Thus, suppose one wants to encode the number $0.42$. Then, one defines the operator 
\begin{equation}
  e^{i\cdot 0.42\widehat{\sigma}_y}\ket{0} = \cos(0.42)\ket{0} -\sin(0.42)\ket{1}
\end{equation}
thus giving a state that have relative phase between $\ket{0}$ and $\ket{1}$.


  








