\section*{Grover's algorithm}
The Grover's algorithm is the first algorithm to mention when discussing sorting. 
The Grover's algorithm is a sorting algorithm that uses a quantum oracle.
We recall that the quantum oracle is a certain function/operator that we consider 
to exist.
In the case of the Grover's algorithm, one defines the Grover's oracle:
\begin{equation}
  U_\omega (x) = 
  \begin{cases}
    \ket{x} \text{ if } x \neq \omega\\
    -\ket{x} \text{ if } x = \omega
  \end{cases}
  \label{eq:first_grovers_oracle}
\end{equation}

We remeber that an oracle is a function that is able to verify some kind of solution.
From the \autoref{eq:first_grovers_oracle}, one can say that the oracle operator corresponds 
to the function
\begin{equation}
  U_\omega\ket{x} = (-1)^{f(x)}\ket{x}, \;\; \text{ with } 
  f(x) = \begin{cases}
    0 \text{ if } x \neq \omega \\
    1 \text{ if } x = \omega
  \end{cases}
  \label{eq:grovers_algorithm}
\end{equation}

Next, one defines the so-called \textsl{Grover's diffusion operator}, sometimes referred to as the 
phase amplification operator. It is defined as 
\begin{gather}
  U_s = 2\ket{s}\bra{s} - \mathbbm{1}
  \label{eq:grovers_diffusion}
\end{gather}
, where $\ket{s}$ is a superposition that will be defined further.

We also define the operator $\widetilde{U}_s$ to simplyfy notation:
\begin{gather}
\widetilde{U}_s = 2\ket{0}^{\otimes n}\bra{0}^{\otimes n} - \mathbbm{1}
\end{gather}

So now, let's consider the circuit itself. We will try to give the mathematical proof of it based on the circuit itself.

\begin{table}[hbt!]
  \centering
  \begin{tblr}{l|r}
    \hspace{-0.2cm}
    \Qcircuit @C=1em @R=1.15em{
      \lstick{\ket{0}}&\qw & \gate{\mathcal{H}} &\multigate{4}{U_\omega}&\gate{\mathcal{H}}& \multigate{4}{\widetilde{U}_s}&\gate{\mathcal{H}}& \qw  & \qw   \\
    \lstick{\ket{0}}&\qw & \gate{\mathcal{H}} & \ghost{U_\omega}        &\gate{\mathcal{H}}& \ghost{\widetilde{U}_s}         &\gate{\mathcal{H}}& \qw  & \qw   \\
    \lstick{\ket{0}}&\qw & \gate{\mathcal{H}} & \ghost{U_\omega}        &\gate{\mathcal{H}}& \ghost{\widetilde{U}_s}         &\gate{\mathcal{H}}& \qw  & \qw   \\
    \vdots           &   &                    & \ghost{U_\omega}        &                  & \ghost{\widetilde{U}_s}         &\vdots            &      & \qw   \\
    \lstick{\ket{0}}&\qw & \gate{\mathcal{H}} & \ghost{U_\omega}        &\gate{\mathcal{H}}& \ghost{\widetilde{U}_s}         &\gate{\mathcal{H}}& \qw  & \qw 
    }
    &\hspace{1.9cm} 
    \Qcircuit @C=1em @R=1.15em{
    \lstick{\ket{0}}&\qw & \gate{\mathcal{H}} &\multigate{4}{U_\omega}  & \multigate{4}{U_s}  & \qw  & \qw   \\
    \lstick{\ket{0}}&\qw & \gate{\mathcal{H}} & \ghost{U_\omega}        & \ghost{U_s}         & \qw  & \qw   \\
    \lstick{\ket{0}}&\qw & \gate{\mathcal{H}} & \ghost{U_\omega}        & \ghost{U_s}         & \qw  & \qw   \\
    \vdots           &   &                    & \ghost{U_\omega}        & \ghost{U_s}         &      & \qw   \\
    \lstick{\ket{0}}&\qw & \gate{\mathcal{H}} & \ghost{U_\omega}        & \ghost{U_s}         & \qw  & \qw 
    }

  \end{tblr}
  \caption{The basic Grover's algorithm step. In order to get better results, i.e. to emphasize the element $\ket{\omega}$ that we're looking for, we would 
  perform multiple of these Grover's steps. Note that there are 2 equivalent cirquits.}
  \label{cirq:grovers_cirq}
\end{table}

On \autoref{cirq:grovers_cirq}, there are 2 fully equivalent circuits. Namely, in the right circuit, the operator 
$\widetilde{U_s}= 2\ket{0}^{\otimes n}\bra{0}^{\otimes n} -\mathbbm{1}$. If we apply to some arbitrary state $\ket{\psi}$ first the Hadamard then the 
$U_s$ then the Hadamard again, we get (we remember that an operator is applied to the right):

\begin{equation}
  \begin{split}
    \ket{\psi}\xrightarrow{\mathcal{H}} \mathcal{H}\ket{\psi} \xrightarrow{U_s} \biggl[2\ket{0}^{\otimes n}\bra{0}^{\otimes n} -\mathbbm{1}\biggr] \mathcal{H}\ket{\psi} = \\
    \biggl[2\ket{0}^{\otimes n}\bra{0}^{\otimes n}\mathcal{H} - \mathcal{H}\biggr] \ket{\psi} \stackrel{\mathcal{H}\ket{0}^{\otimes n}=\ket{s}}{=} \\
    \biggl[2\ket{0}^{\otimes n}\bra{s} - \mathcal{H}\biggr] \ket{\psi} \xrightarrow{\mathcal{H}\text{ again }} \\
    \mathcal{H} \biggl[2\ket{0}^{\otimes n}\bra{s} - \mathcal{H}\biggr] = \biggl[2\mathcal{H}\ket{0}^{\otimes n}\bra{s} - \mathcal{H}\mathcal{H}\biggr] \stackrel{\mathcal{H}^2=\mathbbm{1}}{=}
    2\ket{s}\bra{s}-\mathbbm{1}
\end{split}
\end{equation}

Lets now try to brake down the steps of the algorihtm. As usual, we're preparing the the initial state in the form $\ket{0}^{\otimes n}$. 
The application of the $\mathcal{H}^{\otimes n}$ gives the state $\ket{s} = \frac{1}{\sqrt{N}}\sum_{x=0}^{N-1} \ket{x}$.
Now it's time to apply the Grover's operator. The Grover's operator (as defined in \autoref{eq:grovers_algorithm}) will act on the state $\ket{s}$.

\begin{gather}
  U_\omega \ket{s} = \frac{1}{\sqrt{N}}\sum_{x=0}^{N-1} (-1)^{f(x)}\ket{x} = \frac{1}{\sqrt{N}}\biggl(\sum_{x=0}^{\omega - 1} \ket{x}  
  + \sum_{x=\omega+1}^{N-1} \ket{x} \; \; - \ket{\omega} \biggr) \coloneqq \frac{1}{\sqrt{N}}\ket{\phi} \coloneqq \overline{\ket{\phi}}
  \label{eq:grovers_application}
\end{gather}

%Before applying the diffusion operator, one apply the $\mathcal{H}$ as in \autoref{cirq:grovers_cirq}. 
%We recall an important result \autoref{eq:hadamard_on_generic}. 
%Therefore, one gets for $\mathcal{H}^{\otimes n}$ on $U_\omega \ket{s}$:
%\begin{gather}
%  \mathcal{H}\frac{1}{\sqrt{N}}\ket{\phi} = 
%  \frac{1}{\sqrt{N}}\mathcal{H}^{\otimes n} \sum_{x=0}^{N-1}(-1)^{f(x)}\ket{x} = \\
%  = \frac{1}{N}\sum_{x=0}^{N-1}(-1)^{f(x)} \sum_{y=0}^{N-1}(-1)^{x\cdot y}\ket{y}
%\end{gather}
%Which, in order to simplify the notation and to save space, we'll write as 
%\begin{equation}
%  \sum_{x=0}^{N-1}(-1)^{f(x)} \sum_{y=0}^{N-1}(-1)^{x\cdot y}\ket{y} \equiv 
%  \sum_{x=0}(-1)^{f(x)} \sum_{y=0}(-1)^{x\cdot y}\ket{y} \equiv 
%  \sum_{x,y}(-1)^{f(x)+x\codt y}\ket{y}
%\end{equation}
%Now, it is time to apply the Grover's diffusion operator, defined in 
%\autoref{eq:grovers_diffusion}. So, knowing that the state 






Now it is time to apply the Grover's diffusion operator on the previously obtained state denoted as $\overline{\ket{\phi}}$.

The diffusion operator is defined in \autoref{eq:grovers_diffusion}. We recall that the state $\ket{s}$ is defined as 
$\ket{s} = \frac{1}{\sqrt{N}}\sum_{x=0}^{N-1} \ket{x}$.
The operator is therefore, 

%\begin{equation}
%  \begin{split}
%    U_s \overline{\ket{\phi}} = 2\ket{s}\bra{s} \overline{\ket{\phi}} - \mathbbm{1}\overline{\ket{\phi}} \xrightarrow{\text{ \autoref{eq:grovers_application} }} \\
%    2 \frac{1}{N}\Biggl[ \biggl( \sum_{x=0}^{N-1} (-1)^{f(x)}\ket{x}  \biggr)\cdot \biggl( \sum_{x=0}^{N-1} (-1)^{f(x)}\bra{x}  \biggr) - \mathbbm{1} \Biggr] \ket{\phi} = \\
%    2 \frac{1}{N}\Biggl[ \biggl( \sum_{x=0}^{\omega - 1} \ket{x}  
%  + \sum_{x=\omega+1}^{N-1} \ket{x} \; \; - \ket{\omega} \biggr)\cdot \biggl( \sum_{x=0}^{\omega - 1} \bra{x}  
%  + \sum_{x=\omega+1}^{N-1} \bra{x} \; \; - \bra{\omega} \biggr) - \mathbbm{1} \Biggr] \ket{\phi} = \\
%    2 \frac{1}{N}\Biggl[ \biggl( \sum_{x=0}^{\omega - 1} \ket{x}  
%        + \sum_{x=\omega+1}^{N-1} \ket{x} \; \; - \ket{\omega} \biggr)\cdot \biggl( \sum_{x=0}^{\omega - 1} \bra{x}\ket{\phi}  
%    + \sum_{x=\omega+1}^{N-1} \bra{x}\ket{\phi} \; \; - \bra{\omega}\ket{\phi} \biggr) - \ket{\phi }\Biggr]  
%  \end{split}
%  \label{eq:grovers_apply_diffusion}
%\end{equation}

\begin{equation}
  \begin{split}
    U_s \overline{\ket{\phi}} = 2\ket{s}\bra{s} \overline{\ket{\phi}} - \mathbbm{1}\overline{\ket{\phi}} 
    \frac{1}{\sqrt{N}}\biggl(2\ket{s}\bra{s} \ket{\phi} - \mathbbm{1}\ket{\phi}\biggr) \xrightarrow{\text{ \autoref{eq:grovers_application} }} \\
    \frac{1}{\sqrt{N}}\Biggl[\frac{1}{N}\cdot 2\biggl( \sum_{x=0}^{N-1}\ket{x} \biggr)\cdot \biggl( \sum_{x=0}^{N-1} \bra{x}  \biggr) - \mathbbm{1} \Biggr] \ket{\phi} = \\
    \frac{1}{\sqrt{N}}\Biggl[\frac{1}{N}\cdot 2\biggl( \sum_{x=0}^{N-1}\ket{x} \biggr)\cdot \biggl( \sum_{x=0}^{N-1} \bra{x} \biggr) - \mathbbm{1} \Biggr] \ket{\phi} = \\
    \frac{1}{\sqrt{N}}\Biggl[\frac{1}{N}\cdot 2 \biggl( \sum_{x=0}^{N-1}\ket{x}\biggr)\cdot \biggl( \sum_{x=0}^{N-1} \bra{x} \biggr)\ket{\phi} - \ket{\phi }\Biggr]  
  \end{split}
  \label{eq:grovers_apply_diffusion}
\end{equation}
Let's try to break down the individual components.
The term we want to break down is 
\begin{equation}
\begin{split}
  \biggl( \sum_{x=0}^{N-1} \bra{x} \biggr)\ket{\phi} = \biggl( \sum_{x=0}^{N-1} \bra{x} \biggr)\cdot \biggl( \sum_{x=0}^{N-1}(-1)^{f(x)}\ket{x} \biggr) = \biggl( \sum_{x=0}^{N-1} \bra{x} \biggr)\cdot \biggl( \sum_{y=0}^{N-1}(-1)^{f(y)}\ket{y} \biggr) = \\ 
 = \biggl( \sum_{x=0}^{N-1} \bra{x} \biggr)\cdot \biggl(\sum_{y=0}^{\omega - 1} \ket{y}  
 + \sum_{y=\omega+1}^{N-1} \ket{y} \; \; - \ket{\omega} \biggr) \stackrel{\bra{x}\ket{y} = \delta_{x,y}}{=} (\omega + (N-\omega -1)+1) = N
\end{split}
\end{equation}
The last equality is obtained using the orthogonality and the fact that we're summing from $0$ to $\omega -1$, giving $\omega$ and from $\omega+1$ to $N-1$ giving $N-\omega+1$ 
and the last scalar product gives simply $-1$. Therefore, by plugging the result into \autoref{eq:grovers_apply_diffusion}.
\begin{equation}
  \begin{split}
    \frac{1}{\sqrt{N}}\Biggl[\frac{1}{N}\cdot 2 \biggl( \sum_{x=0}^{N-1}\ket{x}\biggr)\cdot N - \ket{\phi }\Biggr] = \frac{1}{\sqrt{N}}\Biggl[2\biggl( \sum_{x=0}^{N-1}\ket{x}\biggr) - \ket{\phi }\Biggr] 
  \end{split}
\end{equation}
Now, it is quite straightforward to simplify the equation, as we remember that $\ket{\phi}$ differs from $ \sum_{x=0}^{N-1}\ket{x}$ by the factor in front of $\ket{\omega}$. 
We therefore obtain the resulting expression:
\begin{equation}
  \begin{split}
    \frac{1}{\sqrt{N}}\Biggl[2\biggl( \sum_{x=0}^{N-1}\ket{x}\biggr) - \ket{\phi }\Biggr] = \\ 
    \frac{1}{\sqrt{N}}\Biggl[\biggl( \sum_{x=0}^{N-1}\ket{x}\biggr)+ \biggl( \sum_{x=0}^{N-1}\ket{x}\biggr) - \ket{\phi }\Biggr] =  
    \frac{1}{\sqrt{N}}\Biggl[\biggl( \sum_{x=0}^{N-1}\ket{x}\biggr)+ 2\ket{\omega}\Biggr]
  \end{split}
\end{equation}
Now, based on \autoref{cirq:grovers_cirq} left circuit, we're done applying gates and we can now try to measure and try to understand what we get at the output.
We can easily rewrite the last equation as 
\begin{equation}
  \begin{split}
    \frac{1}{\sqrt{N}}\Biggl[\biggl( \sum_{x=0}^{N-1}\ket{x}\biggr)+ 2\ket{\omega}\Biggr] = \\
    \frac{1}{\sqrt{N}}\Biggl[\biggl( \sum_{x=0}^{\omega-1}\ket{x} + \sum_{x=\omega + 1}^{N-1}\ket{x} +\ket{\omega} \biggr)+ 2\ket{\omega}\Biggr] = \\
    \frac{1}{\sqrt{N}}\Biggl[\biggl( \sum_{x=0}^{\omega-1}\ket{x} + \sum_{x=\omega + 1}^{N-1}\ket{x} \biggr)+ 3\ket{\omega}\Biggr] \coloneqq \ket{\eta}
    \end{split}
  \end{equation}
  From that, we can obtain the probability of obtaining the state $\ket{\omega}$ by taking the inner product $|\bra{\eta}\ket{\omega}|^2$. 
  We know that $\bra{x}\ket{\omega}$ will be zero if $\ket{x}\neq \ket{\omega}$. We see that the probability of observing the 
  state $\ket{\omega}$ is greater than any other states. In fact, one can check (I am not covering it here) that by repeating this process more than once, the probability 
  of observing the element $\ket{\omega}$ will be increase. There exists a theorem that states that the number of these kind iterations must be of order $\mathcal{O}(\sqrt{N})$, where 
  $N$ is the number of entry qubits.




%Now, we're ready to proceed further, which is nothing but a simultaneous application of $\mathcal{H}^{\otimes n}$ (as shown in \autoref{cirq:grovers_cirq}).
%So the last operation can be represented by 
%\begin{gather}
%  \mathcal{H}^{\otimes n} \frac{1}{\sqrt{N}}\Biggl[\biggl( \sum_{x=0}^{N-1}\ket{x}\biggr)+ 2\ket{\omega}\Biggr] = \frac{1}{\sqrt{N}}\mathcal{H}^{\otimes n}\biggl(\sum_{x=0}^{N-1}\ket{x}\biggr) + \frac{1}{\sqrt{N}}\mathcal{H}^{\otimes n}2\ket{\omega}
%  \label{eq:grovers_almost_last}
%\end{gather}
%We remember the fact that $\mathcal{H}(\mathcal{H}\ket{\psi}) = (\mathcal{H}\mathcal{H})\ket{\psi} = \mathbbm{1}\ket{\psi} = \ket{\psi}$.
%We also remember that the first operation performed was to apply a $\mathcal{H}^{\otimes n}$ on $\ket{0}^{\otimes n}$. Which gave us 
%$\mathcal{H}^{\otimes n}\ket{0}^{\otimes n} = \frac{1}{\sqrt{N}}\sum_{x=0}^{N-1}\ket{x}$. Therefore in \autoref{eq:grovers_almost_last}, 
%we obtain 
%\begin{gather}
%  \frac{1}{\sqrt{N}}\mathcal{H}^{\otimes n}\biggl(\sum_{x=0}^{N-1}\ket{x}\biggr) + \frac{1}{\sqrt{N}}\mathcal{H}^{\otimes n}2\ket{\omega} = 
%  \ket{0}^{\otimes n} + \frac{1}{\sqrt{N}}\mathcal{H}^{\otimes n}2\ket{\omega}
%\end{gather}
%Now, we're in the position of measuring the states. We can ask ourselves, what is the probability of measuring the state $\ket{omega}$.
%We remember the result \label{eq:hadamard_on_generic}. Which will give us from the previous equality, 
%\begin{gather}
%  \ket{0}^{\otimes n} + \frac{1}{\sqrt{N}}\mathcal{H}^{\otimes n}2\ket{\omega} = \ket{0}^{\otimes n} + 2\frac{1}{N} \sum_{y=0}^{N-1}(-1)^{\omega \cdot y}\ket{y}
%\end{gather}
%Now we ask ourselves what is the probability of measuring $\ket{\omega}$. We suppose that $\ket{\omega} \neq \ket{0}$ (if this is not the case, it is even easier to compute the 
%result).
%So the probability of obtaining $\ket{\omega}$ from this state, will be given with the help of the scalar product:
%\begin{gather}
%  \bra{\omega} \ket{0} + 2\frac{1}{N}\sum_{y=0}^{N-1} (-1)^{\omega\cdot y}\bra{\omega}\ket{y}
%\end{gather}
%
%
%
%
%
%\begin{gather}
%  \bra{\omega}\ket{\phi} = \bra{\omega}\biggl(\sum_{x=0}^{\omega - 1} \ket{x}  
%  + \sum_{x=\omega+1}^{N-1} \ket{x} \; \; - \ket{\omega} \biggr) \stackrel{\bra{a}\ket{b} = \delta_{a,b}}{=} 0 + 0 - \bra{\omega}\ket{\omega} = -1
%\end{gather}
%We're now left with 2 more terms: $\sum_{x=0}^{\omega - 1} \bra{x}\ket{\phi}$ and $\sum_{x=\omega+1}^{N-1} \bra{x}\ket{\phi}$. We remember that 
%
%
%\begin{gather}
%\ket{\phi} = \sum_{x=0}^{N-1} (-1)^{f(x)}\ket{x} =  \sum_{x=0}^{\omega - 1} \ket{x} + \sum_{x=\omega+1}^{N-1} \ket{x} \; \; - \ket{\omega}
%\end{gather}
%Thus the term $\sum_{x=0}^{\omega - 1} \bra{x}\ket{\phi}$ 
%\begin{equation}
%\begin{split}
%  \biggl\langle \sum_{x=0}^{\omega -1}\bra{x} \bigg | \sum_{x=0}^{\omega - 1} \ket{x} + \sum_{x=\omega+1}^{N-1} \ket{x} \; \; - \ket{\omega} \biggl\rangle = \\
%    \biggl\langle \sum_{x=0}^{\omega -1}\bra{x} \bigg| \sum_{x=0}^{\omega - 1} \ket{x} \biggr\rangle + \biggl\langle \sum_{x=0}^{\omega -1} \bra{x} \bigg|\sum_{x=\omega+1}^{N-1} \ket{x} \biggr\rangle 
%    - \biggl\langle \sum_{x=0}^{\omega -1} \bra{x} \bigg| \ket{\omega} \biggr\rangle
%\end{split}
%\end{equation}
%The orthogonality gives that the only non-zero term is $\biggl\langle \sum_{x=0}^{\omega -1}\bra{x} \bigg| \sum_{x=0}^{\omega - 1} \ket{x} \biggr\rangle$. In total, 
%there will be $\omega$ ones. Therefore, this scalar product is simply $\omega$. 
%
%
%The next term is, as said, $\sum_{x=\omega+1}^{N-1} \bra{x}\ket{\phi}$:
%\begin{equation}
%  \begin{split}
%  \biggl\langle \sum_{x=\omega+1}^{N-1}\bra{x} \bigg | \sum_{x=0}^{\omega - 1} \ket{x} + \sum_{x=\omega+1}^{N-1} \ket{x} \; \; - \ket{\omega} \biggl\rangle = \\
%    \biggl\langle \sum_{x=\omega+1}^{N-1}\bra{x} \bigg| \sum_{x=0}^{\omega - 1} \ket{x} \biggr\rangle + \biggl\langle \sum_{x=0}^{\omega -1} \bra{x} \bigg|\sum_{x=\omega+1}^{N-1} \ket{x} \biggr\rangle 
%    - \biggl\langle \sum_{x=\omega + 1}^{N-1} \bra{x} \bigg| \ket{\omega} \biggr\rangle
%  \end{split}
%\end{equation}
%Similarly, as with the previous term, the result will be given instead by $N-1-\omega$, as we're summing from $\omega+1$ to $N-1$.
%Therefore, with all these results, one combines to the \autoref{eq:grovers_apply_diffusion}.
%
%\begin{equation}
%  \begin{split}
%     \frac{1}{N}\Biggl[2 \biggl( \sum_{x=0}^{N-1}\ket{x} \biggr)\cdot \biggl( \sum_{x=0}^{\omega - 1} \bra{x}\ket{\phi}  
%    + \sum_{x=\omega+1}^{N-1} \bra{x}\ket{\phi} \; \; - \bra{\omega}\ket{\phi} \biggr) - \ket{\phi }\Biggr]  = \\
%     \frac{1}{N}\Biggl[2 \biggl(\sum_{x=0}^{N-1}\ket{x} \biggr)\cdot( \omega+(N-1-\omega) + 1) - \ket{\phi }\Biggr]  = 
%     \frac{1}{N}\Biggl[2 \biggl( \sum_{x=0}^{N-1}\ket{x} \biggr)\cdot(N) - \ket{\phi }\Biggr]  
%  \end{split}
%\end{equation}
%
%We can finally develop further the last equality
%\begin{equation}
%  \begin{split}
%     \frac{1}{N}\Biggl[2 \biggl( \sum_{x=0}^{N-1}\ket{x} \biggr)\cdot(N) - \ket{\phi }\Biggr] = 2 \sum_{x=0}^{N-1}\ket{x} - \frac{1}{N}\ket{\phi}
%  \end{split}
%\end{equation}




















