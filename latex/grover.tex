\section*{Grover's algorithm}
The Grover's algorithm is the first algorithm to mention when discussing sorting. 
The Grover's algorithm is a sorting algorithm that uses a quantum oracle.
We recall that the quantum oracle is a certain function/operator that we consider 
to exist.
In the case of the Grover's algorithm, one defines the Grover's oracle:
\begin{equation}
  U_\omega (x) = 
  \begin{cases}
    \ket{x} \text{ if } x \neq \omega\\
    -\ket{x} \text{ if } x = \omega
  \end{cases}
  \label{eq:first_grovers_oracle}
\end{equation}

We remeber that an oracle is a function that is able to verify some kind of solution.
From the \autoref{eq:first_grovers_oracle}, one can say that the oracle operator corresponds 
to the function
\begin{equation}
  U_\omega\ket{x} = (-1)^{f(x)}\ket{x}, \;\; \text{ with } 
  f(x) = \begin{cases}
    0 \text{ if } x \neq \omega \\
    1 \text{ if } x = \omega
  \end{cases}
  \label{eq:grovers_algorithm}
\end{equation}

Next, one defines the so-called \textsl{Grover's diffusion operator}, sometimes referred to as the 
phase amplification operator. It is defined as 
\begin{gather}
  U_s = 2\ket{s}\bra{s} - \mathbbm{1}
  \label{eq:grovers_diffusion}
\end{gather}
, where $\ket{s}$ is a superposition that will be defined further.

So now, let's consider the circuit itself. We will try to give the mathematical proof of it based on the circuit itself.

\begin{table}[ht!]
  \centering
  \begin{tblr}{c}
    \Qcircuit @C=1.2em @R=1.15em{
      \lstick{\ket{0}}&\qw & \gate{\mathcal{H}} &\multigate{4}{U_\omega}&\gate{\mathcal{H}}& \multigate{4}{U_s}&\gate{\mathcal{H}}& \qw  & \qw   \\
    \lstick{\ket{0}}&\qw & \gate{\mathcal{H}} & \ghost{U_\omega}        &\gate{\mathcal{H}}& \ghost{U_s}         &\gate{\mathcal{H}}& \qw  & \qw   \\
    \lstick{\ket{0}}&\qw & \gate{\mathcal{H}} & \ghost{U_\omega}        &\gate{\mathcal{H}}& \ghost{U_s}         &\gate{\mathcal{H}}& \qw  & \qw   \\
    \vdots           &   &                    & \ghost{U_\omega}        &                  & \ghost{U_s}         &\vdots            &      & \qw   \\
    \lstick{\ket{0}}&\qw & \gate{\mathcal{H}} & \ghost{U_\omega}        &\gate{\mathcal{H}}& \ghost{U_s}         &\gate{\mathcal{H}}& \qw  & \qw 
    }
  \end{tblr}
  \caption{The basic Grover's algorithm step. In order to get better results, i.e. to emphasize the element $\ket{\omega}$ that we're looking for, we would 
  perform multiple of these Grover's steps.}
\end{table}

Lets now try to brake down the steps of the algorihtm. As usual, we're preparing the the initial state in the form $\ket{0}^{\otimes n}$. 
The application of the $\mathcal{H}^{\otimes n}$ gives the state $\ket{s} = \frac{1}{\sqrt{N}}\sum_{x=0}^{N-1} \ket{x}$.
Now it's time to apply the Grover's operator. The Grover's operator (as defined in \autoref{eq:grovers_algorithm}) will act on the state $\ket{s}$.

\begin{gather}
  U_\omega \ket{s} = \frac{1}{\sqrt{N}}\sum_{x=0}^{N-1} (-1)^{f(x)}\ket{x} = \frac{1}{\sqrt{N}}\biggl(\sum_{x=0}^{\omega - 1} \ket{x}  
  + \sum_{x=\omega+1}^{N-1} \ket{x} \; \; - \ket{\omega} \biggr) \equiv \frac{1}{\sqrt{N}}\ket{\phi}
  \label{eq:grovers_application}
\end{gather}

Now it is time to apply the Grover's diffusion operator on the previously obtained state denoted as $\ket{\phi}$.
The diffusion operator is defined in \autoref{eq:grovers_diffusion}. We recall that the state $\ket{s}$ is defined as 
$\ket{s} = \frac{1}{\sqrt{N}}\sum_{x=0}^{N-1} \ket{x}$.
The operator is therefore, 
%\begin{equation}
%  \begin{split}
%    U_s \ket{\phi} = 2\ket{s}\bra{s} \ket{\phi} - \mathbbm{1}\ket{\phi} \xrightarrow{\text{ \autoref{eq:grovers_application} }} \\
%    2 \frac{1}{N}\Biggl[ \biggl( \sum_{x=0}^{N-1} (-1)^{f(x)}\ket{x}  \biggr)\cdot \biggl( \sum_{x=0}^{N-1} (-1)^{f(x)}\bra{x}  \biggr) - \mathbbm{1} \Biggr] \ket{\phi} = \\
%    2 \frac{1}{N}\Biggl[ \biggl( \sum_{x=0}^{\omega - 1} \ket{x}  
%  + \sum_{x=\omega+1}^{N-1} \ket{x} \; \; - \ket{\omega} \biggr)\cdot \biggl( \sum_{x=0}^{\omega - 1} \bra{x}  
%  + \sum_{x=\omega+1}^{N-1} \bra{x} \; \; - \bra{\omega} \biggr) - \mathbbm{1} \Biggr] \ket{\phi} = \\
%    2 \frac{1}{N}\Biggl[ \biggl( \sum_{x=0}^{\omega - 1} \ket{x}  
%        + \sum_{x=\omega+1}^{N-1} \ket{x} \; \; - \ket{\omega} \biggr)\cdot \biggl( \sum_{x=0}^{\omega - 1} \bra{x}\ket{\phi}  
%    + \sum_{x=\omega+1}^{N-1} \bra{x}\ket{\phi} \; \; - \bra{\omega}\ket{\phi} \biggr) - \ket{\phi }\Biggr]  
%  \end{split}
%  \label{eq:grovers_apply_diffusion}
%\end{equation}

\begin{equation}
  \begin{split}
    U_s \frac{1}{\sqrt{N}}\ket{\phi} = 2\ket{s}\bra{s} \ket{\phi} - \mathbbm{1}\ket{\phi} \xrightarrow{\text{ \autoref{eq:grovers_application} }} \\
    \frac{1}{\sqrt{N}}\Biggl[\frac{1}{N}\cdot 2\biggl( \sum_{x=0}^{N-1}\ket{x} \biggr)\cdot \biggl( \sum_{x=0}^{N-1} \bra{x}  \biggr) - \mathbbm{1} \Biggr] \ket{\phi} = \\
    \frac{1}{\sqrt{N}}\Biggl[\frac{1}{N}\cdot 2\biggl( \sum_{x=0}^{N-1}\ket{x} \biggr)\cdot \biggl( \sum_{x=0}^{N-1} \bra{x} \biggr) - \mathbbm{1} \Biggr] \ket{\phi} = \\
    \frac{1}{\sqrt{N}}\Biggl[\frac{1}{N}\cdot 2 \biggl( \sum_{x=0}^{N-1}\ket{x}\biggr)\cdot \biggl( \sum_{x=0}^{N-1} \bra{x} \biggr)\ket{\phi} - \ket{\phi }\Biggr]  
  \end{split}
  \label{eq:grovers_apply_diffusion}
\end{equation}
Let's try to break down the individual components.
The term we want to break down is 
\begin{gather}
  \biggl( \sum_{x=0}^{N-1} \bra{x} \biggr)\ket{\phi} = \biggl( \sum_{x=0}^{N-1} \bra{x} \biggr)\cdot \biggl( \sum_{x=0}^{N-1}(-1)^{f(x)}\ket{x} \biggr) = \biggl( \sum_{x=0}^{N-1} \bra{x} \biggr)\cdot \biggl( \sum_{y=0}^{N-1}(-1)^{f(y)}\ket{y} \biggr)
\end{gather}

%\begin{gather}
%  \bra{\omega}\ket{\phi} = \bra{\omega}\biggl(\sum_{x=0}^{\omega - 1} \ket{x}  
%  + \sum_{x=\omega+1}^{N-1} \ket{x} \; \; - \ket{\omega} \biggr) \stackrel{\bra{a}\ket{b} = \delta_{a,b}}{=} 0 + 0 - \bra{\omega}\ket{\omega} = -1
%\end{gather}
%We're now left with 2 more terms: $\sum_{x=0}^{\omega - 1} \bra{x}\ket{\phi}$ and $\sum_{x=\omega+1}^{N-1} \bra{x}\ket{\phi}$. We remember that 
%
%
%\begin{gather}
%\ket{\phi} = \sum_{x=0}^{N-1} (-1)^{f(x)}\ket{x} =  \sum_{x=0}^{\omega - 1} \ket{x} + \sum_{x=\omega+1}^{N-1} \ket{x} \; \; - \ket{\omega}
%\end{gather}
%Thus the term $\sum_{x=0}^{\omega - 1} \bra{x}\ket{\phi}$ 
%\begin{equation}
%\begin{split}
%  \biggl\langle \sum_{x=0}^{\omega -1}\bra{x} \bigg | \sum_{x=0}^{\omega - 1} \ket{x} + \sum_{x=\omega+1}^{N-1} \ket{x} \; \; - \ket{\omega} \biggl\rangle = \\
%    \biggl\langle \sum_{x=0}^{\omega -1}\bra{x} \bigg| \sum_{x=0}^{\omega - 1} \ket{x} \biggr\rangle + \biggl\langle \sum_{x=0}^{\omega -1} \bra{x} \bigg|\sum_{x=\omega+1}^{N-1} \ket{x} \biggr\rangle 
%    - \biggl\langle \sum_{x=0}^{\omega -1} \bra{x} \bigg| \ket{\omega} \biggr\rangle
%\end{split}
%\end{equation}
%The orthogonality gives that the only non-zero term is $\biggl\langle \sum_{x=0}^{\omega -1}\bra{x} \bigg| \sum_{x=0}^{\omega - 1} \ket{x} \biggr\rangle$. In total, 
%there will be $\omega$ ones. Therefore, this scalar product is simply $\omega$. 
%
%
%The next term is, as said, $\sum_{x=\omega+1}^{N-1} \bra{x}\ket{\phi}$:
%\begin{equation}
%  \begin{split}
%  \biggl\langle \sum_{x=\omega+1}^{N-1}\bra{x} \bigg | \sum_{x=0}^{\omega - 1} \ket{x} + \sum_{x=\omega+1}^{N-1} \ket{x} \; \; - \ket{\omega} \biggl\rangle = \\
%    \biggl\langle \sum_{x=\omega+1}^{N-1}\bra{x} \bigg| \sum_{x=0}^{\omega - 1} \ket{x} \biggr\rangle + \biggl\langle \sum_{x=0}^{\omega -1} \bra{x} \bigg|\sum_{x=\omega+1}^{N-1} \ket{x} \biggr\rangle 
%    - \biggl\langle \sum_{x=\omega + 1}^{N-1} \bra{x} \bigg| \ket{\omega} \biggr\rangle
%  \end{split}
%\end{equation}
%Similarly, as with the previous term, the result will be given instead by $N-1-\omega$, as we're summing from $\omega+1$ to $N-1$.
%Therefore, with all these results, one combines to the \autoref{eq:grovers_apply_diffusion}.
%
%\begin{equation}
%  \begin{split}
%     \frac{1}{N}\Biggl[2 \biggl( \sum_{x=0}^{N-1}\ket{x} \biggr)\cdot \biggl( \sum_{x=0}^{\omega - 1} \bra{x}\ket{\phi}  
%    + \sum_{x=\omega+1}^{N-1} \bra{x}\ket{\phi} \; \; - \bra{\omega}\ket{\phi} \biggr) - \ket{\phi }\Biggr]  = \\
%     \frac{1}{N}\Biggl[2 \biggl(\sum_{x=0}^{N-1}\ket{x} \biggr)\cdot( \omega+(N-1-\omega) + 1) - \ket{\phi }\Biggr]  = 
%     \frac{1}{N}\Biggl[2 \biggl( \sum_{x=0}^{N-1}\ket{x} \biggr)\cdot(N) - \ket{\phi }\Biggr]  
%  \end{split}
%\end{equation}
%
%We can finally develop further the last equality
%\begin{equation}
%  \begin{split}
%     \frac{1}{N}\Biggl[2 \biggl( \sum_{x=0}^{N-1}\ket{x} \biggr)\cdot(N) - \ket{\phi }\Biggr] = 2 \sum_{x=0}^{N-1}\ket{x} - \frac{1}{N}\ket{\phi}
%  \end{split}
%\end{equation}




















