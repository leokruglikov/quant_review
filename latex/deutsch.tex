
\theoremstyle{definition}
\newtheorem{definition}{Definition}[section]

\section{Deutsch-Jozsa algorithm}
The Deutsch-Jozsa algorithm was the first algorithm to be proposed.
The goal of the Deutsch's quantum algorithm is to determine a concrete property of a funciton. In this case, we use the quantum 
algorithm, to determine the properties of the function. In the case of the 
\begin{definition}[Constant and balanced]
  The function used here is a function from $\{0,1\}^N \mapsto \{0,1\}$. The outputs of it can be either constant or balanced.
  We say that the function is \textit{constant} if for \underline{any} input the output is constant, that is, whether all ones or all zeroes.
  We say that the function is \textit{balanced} if it outputs ones half of the times and zeroes half of the other times (that is, out of all 
possible inputs, it will output a \verb|1| in the half of the input cases and \verb|0| in the other).
\end{definition}
We will consider the case where $n$ (the number of inputs) is equal to one. That is, the function $f(x)$ has one input bit. $f(x)$, with $x \in \{0,1\}$.
Let's see examples of functions that are either balanced or constant. Let $f_1(x)$. The function happens to output the following: $f_1(0)=0$ and $f_1(1)=0$. We see that the 
function has zeroes for any input. Therefore, we can conclude that the function $f_1$ is constant. Consider now the function $f_2$, satisfying the following:
$f_2(0)=1$ and $f_2(1)=0$. As there are 2 kinds of inputs (\verb|0| and \verb|1|), there are only 2 possible outputs (2 different outputs for 2 different inputs).
We see that for the function $f_2$, half of the outputs (i.e. only one out of two) is \verb|0| and half of the inputs is \verb|1|. Therefore, the function $f_2$ is 
balanced.

The example was a simple case, where the input was simply either \verb|0| or \verb|1|. The definition of this constant/balanced can be extended to a more complex case 
of an input $\{0,1\}^n$, which is the Deutsch-Jozsa algorithm. 
\subsection*{Deutsch's algorithm}
Let's first consider the Deutsch's algorithm \cite{noauthor_deutschjozsa_2022}, that is, with one input bit. Note that in this algorithm we assume or we are promised that the given function is either balanced or constant (it can be neither). The circuit is \textit{postulated} to be.
In order for it to have a working mechanism, we need to prepare the initial states.


\begin{figure}[ht!]
     \begin{subfigure}[b]{0.4\textwidth}
\Qcircuit @C=1em @R=1em{
  \qw & \gate{\mathcal{H}} & \ctrl{1} & \gate{\mathcal{H}} & \qw \\
  \qw & \qw      & \gate{f} & \qw      & \qw \\
}
\caption{Deutsch's circuit}
     \end{subfigure}
     \hfill
     \begin{subfigure}[b]{0.4\textwidth}
\hspace{1cm}
\Qcircuit @C=1em @R=1em{
  \lstick{\ket{0}}           & \qw & \gate{\mathcal{H}}& \ustick{A}\qw  & \ctrl{1} &\ustick{B}\qw & \gate{\mathcal{H}} & \qw \\
  \lstick{\ket{0} - \ket{1}} & \qw & \qw     &\ustick{A}\qw   & \gate{f} & \qw          &\qw      & \qw \\
}

\caption{Initialized states}
     \end{subfigure}
  \caption{Deutsch's circuit}
  \label{cirq:deutsch_start}

\end{figure}

Having the circuit with prepared initial states (\autoref{cirq:deutsch_start}) we're now in the situation in carefully initialize the cirquit. We've identified 2 states of the cirquit: \verb|A| and 
\verb|B|, which we'll quickly discuss \footnote{The $\ket{0}-\ket{1}$ state can be created using the Hadamard gate applied on the \ket{0} state. TODO}.

The first register begins with the state $\ket{0}$. As usual, applying the Hadamard gate $ \mathcal{H} $ on the first register gives us 
$\ket{0} \xrightarrow{\mathcal{H}}{} \frac{1}{\sqrt{2}} ( \ket{0} + \ket{1} )$ as usual. Thus, the total state at the point \textbf{A} is given by the tensor product of both registers, which is $(\ket{0}+\ket{1})\otimes (\ket{0} - \ket{1})$.

After the point \texbf{A} we apply the \textsl{function evaluation} (TODO). Let's consider the rigourous operation:


\begin{math}
  \text{\textmd{state at \textbf{A}}:\ \ \ } \frac{1}{\sqrt{2}}(\ket{0} + \ket{1}) \otimes \frac{1}{\sqrt{2}}(\ket{0}-\ket{1}) \\ 
\end{math}

Now, the state after the function $f$, using the concept of the evaluation function, we have the \textit{general} expression of $\ket{x} \otimes \ket{y} \mapsto \ket{x}\otimes \ket{y \oplus f(x)}$, 
which, in the case of our circuit \autoref{cirq:deutsch_start}, gives the state $\ket{y}=\frac{1}{\sqrt{2}}(\ket{0} - \ket{1}) $ for $\ket{y}$.
Thus, for the circuit between \textbf{A} and \textbf{B}:
\begin{align}
\ket{\psi_B}=\bigg| \frac{1}{\sqrt{2}} (\ket{0}+\ket{1}) \bigg \rangle  &\otimes \bigg| f(x) \oplus \frac{1}{\sqrt{2}} (\ket{0}-\ket{1}) \bigg \rangle \\
\frac{1}{2} \bigg| \ket{0}+\ket{1} \bigg \rangle  &\otimes \bigg| \ket{f(x)}-\ket{1\oplus f(x)} \bigg \rangle \\
\frac{1}{2} \bigg| \ket{0}+\ket{1} \bigg \rangle  &\otimes \bigg| (-1)^{f(x)} (\ket{0}-\ket{1}) \bigg \rangle
\label{eq:deutsch_f(x)}
\end{align}

Now we can expand this further, still for the state at the point B. When we explicit the expression 
in order to see the input of the function $f(x)$. What I mean is that in \autoref{eq:deutsch_f(x)}, the function 
$f(x)$ it accepts the superposition, as the input is an entangled state.

\begin{align}
\ket{\psi_B} = \frac{1}{2} \bigg| \ket{0}+\ket{1} \bigg \rangle  \otimes \bigg| (-1)^{f(x)} (\ket{0}-\ket{1}) \bigg \rangle \\
\ket{\psi_B} = \frac{1}{2} \bigg| (-1)^{f(x)}(\ket{0}+\ket{1})\bigg \rangle \otimes \bigg| \ket{0}-\ket{1} \bigg \rangle \\
\ket{\psi_B} = \frac{1}{2} \bigg| (-1)^{f(0)}\ket{0} + (-1)^{f(1)}\ket{1} \big \rangle \otimes \bigg| \ket{0} - \ket{1} \bigg \rangle
\end{align}
now we apply an unusual trick is not that simple to see but simple to check that it is true. What we will use here is the fact that 
\begin{align}
  f(0)\oplus f(0) \oplus f(1) = f(1)
\end{align}
This can be seen as the term $f(0) \oplus f(0)$ will always give zero and thus, $0 \oplus f(x)=f(x)$ and therefore, 

\begin{align}
\ket{\psi_B} = \frac{1}{2} \bigg| (-1)^{f(0)}\ket{0} + (-1)^{f(0)\oplus f(0)\oplus f(1)}\ket{1} \bigg \rangle \otimes \bigg| \ket{0} - \ket{1} \bigg \rangle \\
\ket{\psi_B} = \frac{1}{2} (-1)^{f(0)}\bigg| \ket{0} + (-1)^{f(0)\oplus f(1)}\ket{1} \bigg \rangle \otimes \bigg| \ket{0} - \ket{1} \bigg \rangle
\label{eq:deutsch_global_phase}
\end{align}
We know however that the global phase do not matter at all. In addition to that, we see that in \autoref{cirq:deutsch_start}, after the point \textbf{B},
we're not interested in the state $\ket{y}$ anymore. Thus, we can simply ignore both the global phase $(-1)^{f(0)}$ and the second state $\ket{0}-\ket{1}$.
Thus, we're left with the state $\ket{\tilde{\psi_B}}\equiv \ket{0} + (-1)^{f(0)\oplus f(1)}\ket{1}$, on which we apply the $\mathcal{H}$ operation.
This gives us 
\begin{align}
  \mathcal{H} \ket{\tilde{\psi_B}} \stackrel{*}{=} \ket{+} + (-1)^{f(0)\oplus f(1)}\ket{-}=\\
  (1+(-1)^{f(0)\oplus f(1)}) \ket{0} + (1-(-1)^{f(0)\oplus f(1)})\ket{1}
  \label{eq:deutsch_f_last}
\end{align}
Therefore, we obtain that if ${f(0)\oplus f(1)}=0$, the measurement of the top qubit will be $\ket{0}$ and if the result ${f(0)\oplus f(1)}=1$, the result of the 
measurement will be given by $\ket{1}$.
Therefore, we will measure the $\ket{0}$ bit if the value of the function is constant (either all ones or all zeroes), and similarly, we will measure the 
value $\ket{1}$ bit if the value of the function is balanced. In both cases, we're making the measurements with probability 1.

\subsection*{Deutsch-Jozsa's algorithm}
The Deutsch-Josza's algorithm is similar to the Deutsch's one. That is, it is a generalization from one bit input ( i.e. $\{0,1\}$) 
to the n bit input (i.e. $\{0,1\}^n$, which is nothing but a bit string).

\begin{table}[!hbt]
  \centering
\begin{tblr}{l || r}
\Qcircuit @C=1em @R=1em{
  \lstick{\ket{0}}&\qw  & \ustick{A}\qw & \gate{\mathcal{H}} & \ustick{B} \qw & \ctrl{1} & \ustick{C} \qw & \gate{\mathcal{H}}& \ustick{D}\qw \\
  \lstick{\ket{0}}&\qw  & \qw & \gate{\mathcal{H}} & \qw & \ctrl{1} & \qw & \gate{\mathcal{H}}& \qw \\
  \lstick{\ket{0}}&\qw  & \qw & \gate{\mathcal{H}} & \qw & \qw & \qw & \gate{\mathcal{H}}& \qw \\
  \lstick{\ket{0}}&\qw  & \qw & \qvdots & & \qvdots & & \qvdots & \\
  \lstick{\ket{0}}&\qw  & \qw & \gate{\mathcal{H}} & \qw & \ctrl{1} & \qw & \gate{\mathcal{H}}& \qw \\
  \lstick{\ket{0}}&\qw  & \qw & \gate{\mathcal{H}} & \qw & \ctrl{1} & \qw & \gate{\mathcal{H}}& \qw \\
  \lstick{\ket{0}-\ket{1}}&\qw & \qw & \qw & \qw & \gate{f} & \qw & \qw & \qw \\
  %\qw & \gate{\mathcal{H}} & \ctrl{1} & \gate{\mathcal{H}} & \qw \\
  %\qw & \qw      & \gate{f} & \qw      & \qw \\
}\hspace{1cm}&
\SetCell[r=9]{} 
\hspace{2cm}
\Qcircuit @C=1em @R=1.5em{
  \lstick{\ket{0}^{\otimes n}} \qw & \ustick{A}\qw & \gate{\Huge\mathcal{H}^{\otimes n}} & \ustick{B} \qw & \ctrl{1} & \ustick{C} \qw & \gate{\Huge\mathcal{H}^{\otimes n}} & \ustick{D}\qw \\
  \lstick{\ket{0}-\ket{1}} & \qw & \qw & \qw & \gate{f} & \qw & \qw & \qw \\
}

\\

\end{tblr}
\caption{The table illustrates 2 schematic circuits for the Deutsch-Josza algorithm. 
Both of them are equivalent. We prefer the second one (the right one) as it is simply shorter.}
\label{table:cirq:deutsch_josza}
\end{table}




The working principle of this generalized version is harder to show and harder to understand. The proof is nicely shown in \cite{noauthor_deutschjozsa_2022}.
Lets start from the initial state, as shown in \autoref{table:cirq:deutsch_josza}. As usual, we sometimes omit the global normalization constant 
(we thus write $\stackrel{*}{=}$).
\begin{align}
  \ket{\psi_A}\stackrel{*}{=}& (\ket{0}\otimes \ket{0} ... \otimes \ket{0})\otimes (\ket{0}-\ket{1}) = \ket{0}^{\otimes n}\otimes (\ket{0}-\ket{1})\\
  \ket{\psi_B}\stackrel{*}{=}& \bigotimes_{i=0}^{n} \mathcal{H} \ket{\psi_A} \stackrel{*}{=} \mathcal{H}^{\otimes n}\ket{0}^{\otimes n}\otimes (\ket{0}-\ket{1})\\
  \stackrel{*}{=}& \ket{+}^{\otimes n} (\ket{0}+\ket{1}) = \sum_{x}^{2^n-1} \ket{x}\otimes (\ket{0}-\ket{1})
\end{align}
Note that $\ket{x}$ here represents a binary string. Indeed, a $\mathcal{H}$ acting on $\ket{0}$ can represent the sum of numbers from 0 to 1. 
($\mathcal{H}\ket{x}\stackrel{*}{=}\ket{0}+\ket{1}$). When acting on a 2D space, 
$(\mathcal{H}\otimes \mathcal{H})(\ket{0}\otimes \ket{0})\stackrel{*}{=}\ket{0}\otimes \ket{0} + \ket{0}\otimes \ket{1} + \ket{1}\otimes \ket{0} + \ket{1}\otimes \ket{1}$.
We therefore see that the Hadamard gate on the n-dimensional space will give the sum of binary strings ranging from $0$ to $2^n-1$. This is what is meant 
in the sum over $\ket{x}$.
Then, the quantum function evaluation, as defined, will create the transformation $\ket{x}\otimes \ket{y} \xrightarrow{f}{} \ket{x}\otimes \ket{f(x)\oplus y}$,
by definition.
Therefore, at the point $C$ (as on \autoref{table:cirq:deutsch_josza}), the transformation give:
\begin{align}
  \ket{\psi_C} &\xleftarrow{f}{} \ket{\psi_B} \\
  \ket{\psi_C} &\stackrel{*}{=} \sum_x^{2^n-1} \ket{x}(\ket{f\oplus 0}-\ket{f \oplus 1}) \\
               &= \sum_x^{2^n-1} \ket{x}(-1)^{f(x)} (\ket{0}-\ket{1})
\label{eq:deutsch_josza_C}
\end{align}
now, we can "ignore" the $\ket{0}-\ket{1}$ part, as after the C point, we're only applying the $\mathcal{H}$ to the top part (i.e. without the 
$\ket{0}-\ket{1}$). We also remember our important result 
\begin{align}
  \matchcal{H} \ket{x} \equiv \bigotimes^n \mathcal{H} \ket{x} = \frac{1}{\sqrt{2^n}} \sum_{y=0}^{2^n-1} (-1)^{x\cdot y}\ket{y}
  \label{eq:hadamard_on_generic}
\end{align}
with the $x \cdot y$ being the vector product between two bitstrings. The bitstrings can be represented as vectors, e.g. $\ket{x}\equiv (0,1,1,0,0,...,1)$ and 
$\ket{y}\equiv (1,0,1,1,...,1)$. Their dot product is defined as $x\cdot y = (x_1 \cdot y_1) \oplus (x_2 \cdot y_2) \oplus (x_3 \cdot x_3) \oplus ... \oplus (x_n \cdot x_n)$.
Therefore, by applying the $\mathcal{H}$ on $\ket{\psi_C}$ given in \autoref{eq:deutsch_josza_C},
\begin{align}
  \bigotimes ^n \mathcal{H} \ket{\psi_C} &\stackrel{*}{=} \mathcal{H} \sum_x^{2^n-1} \ket{x}(-1)^{f(x)} \\
                                         &= \sum_x^{2^n-1} \mathcal{H}\ket{x}(-1)^{f(x)} \stackrel{\text{\autoref{eq:hadamard_on_generic}}}{=} \\
                                         &= \sum_x^{2^n-1} \sum_y^{2^n-1} \ket{y} (-1)^{f(x)}(-1)^{x\cdot y} \\
                                         &= \sum_y^{2^n-1} \Biggl [\; \; \sum_x^{2^n-1} (-1)^{f(x)}(-1)^{x\cdot y} \; \Biggl ] \ket{y}
\end{align}
which is the (almost) final result. This is what we get, when we'll obtain after applying the $\mathcal{H}$ on the $\ket{\psi_C}$ state.
Now, we can see that the $\ket{y}$ are linear combinations of probability amplitudes $\sum_x^{2^n-1} (-1)^{f(x)}(-1)^{x\cdot y}$ for each 
$\ket{y}$. We now have a look at the state $\ket{y}=0$, that is, $\ket{y}=\ket{0,0,...,0}$. The probability of measuring it 
is given by $\bigg| \sum_x^{2^n-1} (-1)^{f(x)} \bigg|^2$. The probability of measuring this $\ket{0}=\ket{y}$ will be given by 1, if the function is constant.
Indeed, if $f(x)$ is constant (either always ones or always zeroes), then we will make the sum of $-1$ if it is always one, and sum of all $(-1)^0$ if always 
zeroes. If it is balanced, then we will make the sum of $(-1)$ and $1$, which will give us 0 probability if the function is balanced.

Thus, we've showed that the algorithm can determine, whether the function is constant or balanced.

\subsubsection*{Example}

The "problem" with this algorithm is that we need to have a very specific oracle. To be more precise, one may thing that we constructed the concept of 
solving the algorithm, based on the given oracle that "comes from nowhere". This is not false. For an arbitrary function, we can't construct the algorithm right 
away.


In order to ckeck the algorithm, one can simply check the algorithm for specific inputs, and whether it represents the specific outputs.



















