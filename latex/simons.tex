\subsection*{Simons algorithm}
Let's now consider the next algorithm, commonly known as the Simon's algorithm or the Simon's 
problem. As usual we will treat the concept mathematically, where it is nicely shown in Wikipedia \cite{noauthor_simons_2023}.

We first want to determine the problem statement. We are given some kind of function $f(x)$
Often, we define this problem through the concept of the whether this function is \textit{one-to-one} or 
\textit{two-to-one}. Both of these notions can be formally related to some function properties, namely surjection 
and injection. Anyways, the \textit{one-to-one} function means that for any input $x$ to the function $f(x)$, there's 
only one possible output. The two-to-one means that there's possibly several inputs for one single output. 
For example, $f(0)=0, f(1)=1, f(2)=2, f(3)=3$ - which is a one-to-one function. Now, an example of a two-to-one function is 
given by $f(0)=f(1)=0, f(2)=f(3)=1$. Note that for the inputs, they can be represented as binary strings, representing numbers.

One can define it in another way, which is fully equivalent. Indeed, the problem now goes as follows:
we're given a function $f(x)$ and the number/binary string $s$. We're now given a promise for $f(x)$:
given $f(x)$ and $s$, $f(x)=f(y)$ if and only if $x\oplus y \in \{0^{\otimes n}, s\}$. That is, the output 
will be same for two different inputs if and only if $x\oplus y$ is either $0^{\otimes n}$ or $s$. Here, we define 
the operation $x \oplus y$ to be the bitwise XOR. That is, $x\oplus y \equiv (x_1 \text{ XOR } y_1, x_2 \text{ XOR } y_2, ..., x_n \text{ XOR } y_n)$.

\begin{wraptable}{l}{7cm}
  \begin{tblr}{c}
    \hspace{1cm}
    \Qcircuit @C=0.75em @R=1em {
      \lstick{\ket{0}^{\otimes n}} & \qw & \gate{\mathcal{H}^{\otimes n}} & \qw & \multigate{1}{f} & \qw & \gate{\mathcal{H}^{\otimes n}} \qw &\qw \\
      \lstick{\ket{0}^{\otimes n}} & \qw & \qw                            & \qw & \ghost{f}         & \qw & \qw & \qw 
    }
  \end{tblr}
\end{wraptable}
\label{cirq:simons}

The goal of the algorithm is to determine the bitstring $s$. To schematically show that these two formulations (in terms of one-to-one \& two-to-one) 
and the last one are equivalent, we first note that $a \oplus b = 0^{\otimes n} \iff a=b$.
Another fact that we notice is the following: for some $x$ and $s$, we have that $x\oplus y = s$ is unique for $x$ if and only if $s \neq 0^{\otimes n}$. 
In other words, the output of the operation $s$ is uniquely determined by the inputs only if $s=0^{\otimes n}$. Therefore, 
we say that if $s\neq 0^{\otimes n}$, the function is two-to-one and if $s=0^{\otimes n}$, the function is one-to-one.

Let's now consider the circuit of the algorithm. The circuit is given in \autoref{cirq:simons}.
As usual, we're considering the initial state, which will go through the first $\mathcal{H}$ and we'll obtain the 
usual result 

\begin{align}
  \mathcal{H}^{\otimes n} \ket{0}^{\otimes n}=\frac{1}{\sqrt{2^n}}\sum_{x=0}^{2^n-1} \ket{x}
\end{align}

Then we will apply our quantum function evaluation to the state on both registers:

\begin{align}
  \sum_{x=0}^{2^n-1} \ket{x} \ket{0}^{\otimes n} \xrightarrow{f(x)} \sum_{x=0}^{2^n-1} \ket{x}\ket{f(x)\oplus 0} =\sum_{x=0}^{2^n-1} \ket{x}\ket{f(x)} 
\end{align}

Which is nothing but the state after the function evaluation and before the second $\mathcal{H}^{\otimes n}$ on the first register. 
We will then obtain as usual:
\begin{align}
  &\sum_{x=0}^{2^n-1} \ket{x}\ket{f(x)} \xrightarrow{(\mathcal{H}^{\otimes n})\otimes (\mathbbm{1}^{\otimes n}) } \\ 
  &\sum_{x=0}^{2^n-1} \biggl[ \sum_{y=0}^{2^n-1} (-1)^{x\cdot y}\ket{y} \biggr] \ket{f(x)}=\sum_{y=0}^{2^n-1}\ket{y} \biggl[ \sum_{x=0}^{2^n-1}(-1)^{x\cdot y}\ket{f(x)} \biggr]
\end{align}

which is our state that will be measured. In fact, as usual, we've simplified the multiplicative factor. Therefore, the "correct" state that we'll be 
measuring, will be as the one described above times the factor $\nicefrac{1}{(2^n)}$. Therefore for a certain $\ket{y}$, the probability of this 
happening will be given by 
\begin{align}
  \Bigg| \Bigg| \frac{1}{2^n} \sum_{x=0}^{2^n-1} (-1)^{x\cdot y} \ket{f(x)} \Bigg| \Bigg|^2
  \label{eq:simons_last_meas}
\end{align}
Now we consider the 2 possible cases (whether $f$ will be one-to-one or two-to-one). 

If the function $f(x)$ is one-to-one, the \autoref{eq:simons_last_meas} will give us the probability of measuring some ket $\ket{y}$.
An another way to write this probability in \autoref{eq:simons_last_meas}, we can write the probability for a state $\ket{y}$ to be measured will be 
given by $|\bra{y}\ket{x}|$. Therefore, we have that the probability is given by nothing but 
\begin{gather}
  \Bigg< \frac{1}{2^n} \sum_{x=0}^{2^n-1} (-1)^{x\cdot y}\ket{f(x)} \Bigg| \frac{1}{2^n} \sum_{x'=0}^{2^n-1}(-1)^{x' \cdot y} \ket{f(x')} \Biggr> \\
  (\frac{1}{2^n})^2 \sum_{\substack{x=0,x'=0\\ x,x'}} (-1)^{x\cdot y} (-1)^{x' \cdot y} \bra{f(x)}\ket{f(x')}, \text{ using that } \bra{a}\ket{a'}=\delta_{a,a'} \\
  (\frac{1}{2^n})^2 ((-1)^{0\cdot y} (-1)^{0\cdot y} \bra{f(0)}\ket{f(0)} + (-1)^{0\cdot y} (-1)^{1\cdot y} \bra{f(0)}\ket{f(1)} + ... + (-1)^{2^n-1\cdot y} (-1)^{2^n-2\cdot y} \bra{f(2^n-1)}\ket{f(2^n-2)} + (-1)^{2^n-1\cdot y} (-1)^{2^n-1\cdot y} \bra{f(2^n-1)}\ket{f(2^n-1))
\end{gather}
We know that if $\bra{f(x_i)}\ket{f(x_j)} == 1$ if $i == j$, therefore we know that $2^n$ of the terms will be zero. Therefore, we have that 
$(2^n-1)(2^N-1) - 2^n$ terms will be zero. Therefore, we have that the probability of measuring $\ket{y}$ will be given by
\begin{gather}
  We know that if $\bra{f(x_i)}\ket{f(x_j)} == 0$ if $i == j$ \\
   \frac{1}{2^{2n}} \sum_{x=0}^{2^n-1}(-1)^{2x\cdot y}= \frac{1}{2^{2n}} (2^n)\cdot(1)=\frac{1}{2^n}
   \label{eq:simons_inner_product}
\end{gather}
Which is not surprising, as since $f$ is one-to-one, we're going through all the basis vectors.

Now we consider the second case. That is, the case, where the function is not one-to-one. We can follow the nice trick shown 
in Wikipedia \cite{noauthor_simons_2023}. When the function is not one-to-one, this means therefore that for some inputs $x_1$ \& $x_2$,
we have the same outputs $f(x_1)=f(x_2)=z$, where $z$ is some kind of value in the range/domain of the function $f$. For the specific case, one may write for 
the probability for some chosen $\ket{y}$ as for \autoref{eq:simons_last_meas}.
\cite{noauthor_simons_2023}.
\begin{gather}
  \Bigg|\Bigg| \frac{1}{2^n}\sum_{x=0}^{2^n-1} (-1)^{x\cdot y} \ket{f(x)} \Bigg|\Bigg|^2 = 
  \Bigg|\Bigg| \frac{1}{2^n} \sum_{z\in \text{Range}(f)}((-1)^{y\cdot x_1} +(-1)^{y\cdot x_2})\ket{z} \Bigg| \Bigg|^2
  \label{eq:simons_z_range}
\end{gather}
\begin{wraptable}{l}{3cm}
  \centering
  \begin{tabular}{|c|c|c|}
    \hline
    $x_1$ & $x_2$ & $s$ \\
    \hline
    \hline
    0 & 0 & 0 \\
    \hline
    1 & 0 & 1 \\
    \hline
    0 & 1 & 1 \\
    \hline
    1 & 1 & 0 \\
    \hline
  \end{tabular}
  \caption{Truth table for XOR.}
  \vspace{-1.0cm}
  \label{table:xor_truth_table}
\end{wraptable}

Note that here we've changed the sum over $x$ i.e. over sum of the domain, to the sum over the range of its values $z$.
We now note another important property for the XOR operation. The truth table of XOR is given by the \autoref{table:xor_truth_table}.
From which we clearly see that $x_1 \otimes x_2 = s \iff x_1 \otimes s = x_2$. Therefore, if this is valid for one bit string, it 
will also be valid for any bit strings of any length, as the bitwise XOR only operates on pairs of bits.
Thus, by observing the property of $x_1 \otimes x_2 = s \iff x_1 \otimes s = x_2$, we can write the \autoref{eq:simons_z_range} 
as
\begin{gather}
  \Bigg|\Bigg| \frac{1}{2^n} \sum_{z\in \text{Range}(f)}((-1)^{y\cdot x_1} +(-1)^{y\cdot x_2})\ket{z} \Bigg| \Bigg|^2 =\\
  =\Bigg| \Bigg| \frac{1}{2^n} \sum_{z\in \text{Range}(f)}((-1)^{y\cdot x_1} +(-1)^{y\cdot (x_1 \otimes s)})\ket{z} \Bigg| \Bigg|^2 \\
  =\Bigg| \Bigg| \frac{1}{2^n} \sum_{z\in \text{Range}(f)}((-1)^{y\cdot x_1} +(-1)^{y\cdot x_1 \otimes y\cdot s)})\ket{z} \Bigg| \Bigg|^2 \\
  =\Bigg| \Bigg| \frac{1}{2^n} \sum_{z\in \text{Range}(f)} (-1)^{y\cdot x_1}(1+(-1)^{y\cdot s})\ket{z} \Bigg| \Bigg|^2 
\end{gather}
From where, we can observe some things: if the chosen $y$ happens to be such that $y\cdot s = 1$, then the probability (the sum given right above)
will be given by 0 (the factor in front of $\ket{z}$ will be 0). Now, if the value of the product $y\cdot s = 0$, it will not be zero.
instead, the sum becomes of the form 

\begin{gather}
  \Bigg| \Bigg| \frac{1}{2^n} \sum_{z\in \text{Range}(f)} (-1)^{y\cdot x_1}(2)\ket{z} \Bigg| \Bigg|^2 \\
  \Bigg| \Bigg| \frac{1}{2^{n-1}} \sum_{z\in \text{Range}(f)} (-1)^{y\cdot x_1}\ket{z} \Bigg| \Bigg|^2 
\end{gather}
Using the same trick as in \autoref{eq:simons_inner_product}, we can deduce that the probability will be given by $\frac{1}{2^{n-1}}$.

From these two cases, we can therefore deduce the following: the probability of measuring a certain $y$ if $s=0^{\otimes n}$ (the first case) will be given 
by $\frac{1}{2^n}$.
And the probability for a certain $y$ if $s\neq 0^{\otimes n}$, we have 2 (sub)cases. That is, depending on whether this $y$ will either obey
$y\cdot s=0$ or $y\cdot s\neq 0$. If it is the case of $y\cdot s=0$, we will have 0 probability of measuring this $\ket{y}$. If, on the other hand,
$y\cdot s\neq 0$, it will be given by $\frac{1}{2^{n-1}}$. 


Therefore, we deduce that in both cases, for some measured y, we will have that \underline{$y\cdot s =0$} (as in the first case, $s=0$ by definition and in the second, 
if it is not the case, the probability of measuring the $y$ state will be zero).

During the mathematical process, starting from \autoref{eq:simons_last_meas}, we considered some specific $\ket{y}$ that we'll get at the end/output.
So at the end we'll get several $\ket{y}$'s with probability, depending on different cases, as discussed above. The key point here is that 
these states, i.e. all $y$'s obey $y\cdot s = 0$.
As provided in \cite{noauthor_simons_2023}, we use the classical post processing in order to obtain the necessary $s$ as required in the problem statement.
It is important to note that, it may happen that there are "missing states", as there are sometimes 0 probability of observing some variables. 











